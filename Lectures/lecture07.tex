\ProvidesFile{lecture07.tex}[Lecture 7]


\section{Fields}

\subsection{Characteristic}

\begin{definition}
Let $F$ be a field.
The characteristic of $F$ is the minimial positive integer $p$ such that
\[
\underbrace{1+\ldots + 1}_p = 0
\]
If there is no such $p$ the characteristic is said to be zero.
The characteristic of $F$ is denoted by $\chr F$.
\end{definition}

I want to introduce a convenient notation, if we add an element $x\in F$ $n$ time, where $n\in \mathbb N$, we may denote the sum as follows
\[
n x = \underbrace{x+\ldots + x}_n
\]
In particular, the characteristic of $F$ is the smallest positive integer $p$ such that $p \cdot 1 = 0$.
\begin{examples}
\begin{enumerate}
\item If $F = \mathbb Q$, then the sum $1+\ldots + 1$ is never zero.
Hence, $\chr \mathbb Q = 0$.

\item If $F = \mathbb Z_p$, then $p$ is the smallest positive integer such that $1 + \ldots + 1 = p \cdot 1 = 0$ in $\mathbb Z_p$.
Hence, $\chr \mathbb Z_p = p$.
\end{enumerate}
\end{examples}

\begin{claim}
If $F$ is a field, then $\chr F$ is either $0$ or a prime number.
\end{claim}
\begin{proof}
Suppose the characteristic is not zero.
Assume that $\chr F = p = n m$ is not prime.
Then
\[
0 = \underbrace{1+\ldots + 1}_{nm} = (\underbrace{1+\ldots + 1}_{n})(\underbrace{1+\ldots + 1}_{m}) = (n \cdot 1) (m \cdot 1)
\]
Moreover, the elements $n\cdot 1$ and $m\cdot 1$ are nonzero because $p = nm$ was the smallest one.
But the product of the elements $n\cdot 1$ and $m \cdot 1$ is zero.
This contradicts to $F$ being a field.
\end{proof}

\begin{remark}
Suppose $F$ is a field.
We have a unique ring homomorphism $\phi\colon \mathbb Z\to F$ by the rule $n \mapsto n \cdot 1$.
Then the kernel of this homomorphism is an ideal of $\mathbb Z$.
Every ideal of the ring is of the form $(p)$ for some $p\in \mathbb N$ (see Claim~\ref{claim::ZIdeals}).
In this case, $p$ equals the characteristic of $F$.
This explains the relation between the characteristic of a field and the ideals of the ring of integers.
\end{remark}

\begin{claim}
\label{claim::FieldIdeals}
Let $R$ be a commutative ring.
Then, $R$ is a field if and only if there are only two ideals $0$ and $R$.
\end{claim}
\begin{proof}
Suppose that $R$ is a field and $I\subseteq R$ is an ideal.
If $I = 0$, then we are done.
We may assume that $I$ contains a nonzero element and we should show that $I = R$.
Let $x\in I$ be a nonzero element.
Since $R$ is a field, there exists $x^{-1}\in R$.
Since $I$ is an ideal $1 =x^{-1}x\in I$.
Hence, for any $y\in R$, the element $y = y 1 \in I$.

Now, suppose that $R$ contains only trivial ideals $0$ and $R$.
Let $x\in R$ be a nonzero element.
Then the set $I = \{rx \mid r\in R\}$ is an ideal of $R$.%
\footnote{Here, it is important that $R$ is commutative.}
Since there are only two ideals $0$ and $R$, $I$ must be one of them.
Since $I$ contains a nonzero element, $I$ coincides with $R$.
In particular, $1\in I$, that is $1 = rx$ for some $r\in R$.
Since $R$ is commutative, $x$ is invertible and we are done.
\end{proof}

The latter claim means that we should study fields elements-wise because ideals distinguish nothing in this case.
However, ideals somehow measure the difference between an arbitrary ring and a field.
Less ideals a ring has, the closer the ring is to a field.

\begin{claim}
\label{claim::ZpField}
A ring $\mathbb Z_n$ is a field if and only if $n$ is prime.
\end{claim}
\begin{proof}
This is an immediate consequence of Claim~\ref{claim::ZnIdeals} and Claim~\ref{claim::FieldIdeals}.
\end{proof}


\begin{claim}
\label{claim::PrimeSubfieldCharP}
Suppose $F$ is a field of prime characteristic $p$.
Then $F$ contains $\mathbb Z_p$ as a subfield.
\end{claim}
\begin{proof}
The field $F$ contains the identity element $1\in F$.
Let us consider the set $\{0, 1, 2\cdot 1, \ldots , (p-1)\cdot 1\}\subseteq F$.
Then there is an obvious bijection between $\mathbb Z_p$ and the subset.
We should show that the bijection is an isomorphism of rings.
In order to do that, we need to show that
\[
n\cdot 1 + m \cdot 1 = (n+m\!\!\mod p)\cdot 1,\quad
(n\cdot 1)(m\cdot 1) = (mn\!\!\mod p)\cdot 1
\]
Indeed, if $m+n = qp + r$, then
\[
n\cdot 1 + m \cdot 1 = (m+n)\cdot 1 = (qp) \cdot 1 + r\cdot 1 = q(p \cdot 1) + r\cdot 1 = r\cdot 1
\]
The second statement is shown in a similar way.
\end{proof}

\subsection{Field extension}

Suppose $F$ is a field and $K$ is another field containing $F$ as a subfield.
Then we will say that $K$ is an extension of $F$.
In this case, $K$ may be considered as a vector space over $F$.
Hence, there is a dimension of $K$ over $F$.
The dimension of $K$ over $F$ is called the degree of $K$ over $F$ and is denoted by $[K:F] = \dim_F K$.


We need a method to produce fields.
The following claim explains one of the most useful constructions.

\begin{claim}
\label{claim::PolyQuotField}
Let $F$ be a field, $f\in F[x]\setminus F$ be a polynomial.
The ring $F[x]/(f)$ is a field if and only if $f$ is irreducible nonzero polynomial.
\end{claim}
\begin{proof}
Suppose that $f$ is reducible.
Then $f = gh$, where $\deg g < \deg f$ and $\deg h < \deg f$.
Then $h\neq 0$ in $F[x]/(f)$ as well as $g\neq 0$ in $F[x]/(f)$.
But $gh = f = 0$ in $F[x]/(f)$.
Hence, $g$ and $h$ are nonzero zero divisors.
But zero divisors are not invertible.
The latter contradicts to the definition of a field.

If $f$ is irreducible we should show that any nonzero $g\in F[x]/(f)$ is invertible.
Since $\deg g < \deg f$ and $g\neq 0$, $f$ and $g$ are coprime.
Hence, $1 = (g, f)$.
By Claim~\ref{claim::PolyGCD} item~(3), it follows that $1 = ug + vf$ for some $u,v\in F[x]$.
But the latter means that $1 = ug\pmod f$ and hence, $u = g^{-1}$ in $F[x]/(f)$.
\end{proof}

\begin{remarks}
\begin{itemize}
\item It is also worth mentioning that the element $x\in F[x]/(p)$ is a root of the polynomial $p$ in the field $F[x]/(p)$.
Indeed, $p(x) = 0$ in $F[x]/(p)$ be definition.

\item Let us consider the field of real numbers $\mathbb R$.
Then the polynomial $x^2 + 1\in \mathbb R[x]$ is irreducible.
Hence $\mathbb R[x]/(x^2 +1)$ is a field and the element $x$ becomes a root of $x^2 + 1$.
Explicitly elements of $\mathbb R[x]/(x^2 + 1)$ are of the form $a + bx$, where $a, b\in \mathbb R$ and we also know that $x^2 = -1$.
Hence, this is the usual model for the complex numbers.

\item If $\deg f = n$, then the elements $1, x, \ldots, x^{n-1}$ form a basis of $F[x]/(f)$ over the field $F$.
In particular, $\dim_F F[x]/(f) = \deg f$.
\end{itemize}
\end{remarks}


\paragraph{Extension by a root}

Here I am discussing a standard way to produce a larger field containing a root of a given polynomial.
Suppose $F$ is a field and $f\in F[x]$ is a polynomial.
Now, I want to produce a field $L$ containing $F$ and an element $\alpha \in L$ such that $f(\alpha) = 0$.
If we are lucky enough and $F$ already has such an element, then there is nothing to do.
However, the problem is not trivial in case there is no such element inside $F$.
We already know (see Claim~\ref{claim::PolyUFD}) that $f$ is a product of irreducible polynomials, that is $f = p_1\ldots p_n$ ($p_i$ need not be different here).
It is enough to solve the problem for some $p_i$ (if we find an extension $L$ containing a root of $p_1$, then it will be an extension containing a root of $f$).
Hence, we may assume that $f$ is irreducible.

Let us define $L$ to be the ring of remainders modulo $f$, that is $L = F[x]/(f)$.
By Claim~\ref{claim::PolyQuotField}, $L$ is a field.
The element $x\in F[x]/(f)$ will be denoted by $\alpha$.
Let us show that $\alpha$ is a root of $f$.
Indeed, if we consider $f(\alpha)$ in $L$, we have
\[
f(\alpha) = f(x) = 0 \pmod f
\]
That is $f(\alpha) = 0$ in $L$.
It should be noted that the degree of $L$ over $F$ coincides with $\deg f$.


\subsection{Finite fields}

Now, I want to deal with finite field only, that is with fields consisting of finitely many elements.

\begin{claim}
If $F$ is a finite field, then its characteristic is not zero.
In particular, it contains $\mathbb Z_p$ for $p = \chr F$.
\end{claim}
\begin{proof}
If $F$ is finite, then the ring homomorphism $\mathbb Z \to F$ by the rule $n \mapsto n \cdot 1$ cannot be injective.
Hence its kernel is of the form $(p)$ for some nonzero $p$.
The last statement follows from Claim~\ref{claim::PrimeSubfieldCharP}.
\end{proof}

\begin{claim}
Suppose $F$ is a finite field and $\chr F = p$.
Then, $|F| = p^n$, where $n = [F:\mathbb Z_p]$.
\end{claim}
\begin{proof}
By Claim~\ref{claim::PrimeSubfieldCharP}, $F$ contains $\mathbb Z_p$ as a subfield.
Now, we consider $F$ as a vector space over $\mathbb Z_p$.
Since $F$ is finite, it has finite dimension.
Therefore, $F$ is isomorphic to $\mathbb Z_p^n$ as a vector space.
But, $\mathbb Z_p^n$ has exactly $p^n$ elements, where $n$ is the dimension of $F$ over $\mathbb Z_p$.
\end{proof}

\begin{claim}
Let $F$ be a finite field.
Then the group $F^*$ is cyclic of order $|F| - 1$.
\end{claim}
\begin{proof}
The group $F^*$ is a finite abelian group.
Hence, it is isomorphic to $\mathbb Z_{d_1}\times \ldots \times \mathbb Z_{d_k}$, where $d_1|\ldots |d_k$ (see Claim~\ref{claim::FAGClass}).
In particular, for every element of the group $F^*$ we have $x^{d_k} = 1$.
Thus, all elements of the group $F^*$ are the roots of the polynomial $x^{d_k}-1$.
Because of the unique factorization property, each polynomial $f$ has at most $\deg f$ roots.
Hence, $|F^*|\leqslant d_k$.
From the other hand $|F^*| = d_1\ldots d_k$.
The only way this is possible is if $d_1 = \ldots =d_{k-1} = 1$.
That is $F^*$ is isomorphic to $\mathbb Z_{d_k}$.
\end{proof}

There is a very important classification result (I am not going to prove it).

\begin{claim}
For any prime number $p$ and any positive integer $n$, there exists a unique (up to isomorphism) field $F$ consisting of $p^n$ elements.
\end{claim}

Since there is a unique field of $p^n$ elements, it has a special name $\mathbb F_{p^n}$.
In particular, $\mathbb F_p = \mathbb Z_p$.
However, it should be noted that $\mathbb F_{p^n}\not\simeq\mathbb Z_{p^n}$, for example, because $\mathbb Z_{p^n}$ contains zero divisors.

A good question is how to produce all finite fields.
Suppose we want to produce a field $F$ such that $|F| = p^n$ for some prime $p$ and positive integer $n$.
A starting point is the field $\mathbb F_p = \mathbb Z_p$.
We should find an irreducible polynomial $f\in \mathbb Z_p[x]$ of degree $n$.
Then, the required field is $\mathbb F_{p^n} = \mathbb Z_p[x]/(f)$.
It should be noted that there could be many different irreducible polynomials of degree $n$ over $\mathbb Z_p$.
However, the resulting fields will be isomorphic.
How to choose $f$?
Simply take anyone or the most convenient one for your purposes.

\begin{example}
Let us produce $\mathbb F_4$.
The base field is $\mathbb Z_2$.
There is only one irreducible polynomial of degree $2$, that is $x^2 + x + 1 \in \mathbb Z_2[x]$.
Then the required field is 
\[
\mathbb F_4 = \mathbb Z_2[x]/(x^2 + x + 1) = \{a + b x\mid a, b\in \mathbb Z_2\}
\]
The addition is given by the coordinate-wise addition.
In order to compute a product of any two elements of $\mathbb F_4$, it is enough to compute products of all powers of $x$.
The products $1\cdot 1 = 1$ and $1\cdot x = 1$ are simple to compute.
Now, we need to compute $x\cdot x$.
It equals $x^2 = 1 + x  \pmod{x^2 + x + 1}$.
In general, a product is computed like this
\[
(a+bx)(c+d x) = ac + ad x + bc x + bd x^2 = ac + ad x + bc x + bd (1+x) = ac + bd + (ad + bc + bd) x
\]
\end{example}

\begin{remarks}
\begin{itemize}
\item 
The field $\mathbb Z_p$ is contained in $\mathbb F_{p^n}$.
In particular, the group $\mathbb Z_p^*$ is contained in the group $\mathbb F_{p^n}^*$.
As we mentioned already in Section~\ref{section::DiscreteLog}, the discrete logarithm problem is hard to solve in $\mathbb Z_p^*$.
This implies that the discrete logarithm problem must be hard for $\mathbb F_{p^n}^*$.
Indeed, if it were simple to solve $g^x = h$ for $x$, when we are given $g,h \in \mathbb F_{p^n}^*$, we could take $g, h\in \mathbb Z_p^* \subseteq \mathbb F_{p^n}^*$.
This would imply that the discrete logarithm problem for $\mathbb Z_p^*$ were simple.
The immediate consequence of this observation is that we can use $\mathbb F_{p^n}^*$ in the Diffie-Hellman approach discussed before.

\item
If we want to use $\mathbb F_{p^n}^*$ in cryptography, we have to find a generator of the group.
If we produced the field in the form $\mathbb Z_p[x]/(f)$ for some irreducible $f\in \mathbb Z_p[x]$ of degree $n$, a good candidate for the generator of $\mathbb F_{p^n}^*$ would be $x$.
However, the element $x$ of $\mathbb F_{p^n}^*$ need not be a generator.
It depends on the choice of $f$.

For example, if we take $\mathbb Z_3$, there are three irreducible polynomials of degree $2$: $f_1 = x^2+1$, $f_2 = x^2 + x - 1$, $f_3 = x^2 - x - 1$.
If we use the first polynomial, we get $\mathbb F_9 = \mathbb Z_3[x]/(x^2 + 1)$.
The group $\mathbb F_9^* \simeq \mathbb Z_8$, hence the generator must have order $8$.
However, $x^4 = (x^2)^2=(-1)^2 = 1$ in this case.
Thus, the order of $x$ is $4$ and it is not a generator.
From the other hand, we can use $f_2$ and get $\mathbb F_9 = \mathbb Z_3[x]/(x^2+x-1)$ or $f_3$ and get  $\mathbb F_9 = \mathbb Z_3[x]/(x^2-x-1)$.
In both cases, a direct computation shows that $x$ is a generator of $\mathbb F_9^*$.
\end{itemize}
\end{remarks}

\subsection{Galois random generator}

There is a notion of Linear-feedback shift register.
This is a scheme to generate random numbers.
It is usually presented in two different forms: Fibonacci and Galois.
They are dual to each other in some sense that I am not going to explain here.
The Galois scheme is a bit more complicated but we can easily describe it using the language of finite fields.

Suppose we have a finite alphabet $\mathbb Z_p = \{0,1,\ldots,p-1\}$ and we want to produce a random sequence of length $n$ consisting of these elements, that is we want randomly choose an element of $\mathbb Z_p^n$.
A key observation here is that we can identify $\mathbb Z_p^n$ with $\mathbb F_{p^n}$ and choose randomly an element of a finite field instead.
Let me describe the whole process.


Suppose $p$ is a prime number and $n$ is a positive integer.
We choose an irreducible polynomial $f\in \mathbb Z_p[x]$ of degree $n$ and produce $\mathbb F_{p^n} = \mathbb Z_p[x]/(f)$.
By definition
\[
\mathbb F_{p^n} = \{a_0 + a_1 x + \ldots + a_{n-1}x^{n-1}\mid a_i\in \mathbb Z_p\}
\]
Hence, every element of the field is described as a sequence $(a_0, a_1,\ldots,a_{n-1})$.
Now, we choose a generator $g\in \mathbb F_{p^n}^*$.
This is an unpleasant step but we have to make it only once.
Usually, we choose $f$ in such a way that $x$ is a generator and take $g$ to be $x$.
Now, we generate a sequence $g, g^2, g^3, g^4,\ldots$.
Each power of $g$ corresponds to a sequence of coefficients as above.
We can also start from some fixed power of $g$, that is, we fix a power $k\in \mathbb N$, and generate a sequence $g^k, g^{k+1}, g^{k+2}, \ldots$.
In practice, we usually choose a some $h\in \mathbb F_{p^n}^*$ and generate $hg, hg^2, hg^3, hg^4,\ldots$.
Since $h = g^k$ for some $k$, this is exactly the same approach.

\paragraph{Matrix form}

As before, we assume that
\[
\mathbb F_{p^n} = \mathbb Z_p[x]/(f)= \{a_0 + a_1 x + \ldots + a_{n-1}x^{n-1}\mid a_i\in \mathbb Z_p\}
\]
In this case, $1, x, x^2,\ldots,x^{n-1}$ is a basis of $\mathbb F_{p^n}$ over $\mathbb Z_p$.
Suppose, that the polynomial $f = x^n + c_{n-1}x^{n-1} +\ldots + c_1 x + c_0$ is chosen in such a way that $x$ is a generator of $\mathbb F_{p^n}^*$.
The map $\phi \colon \mathbb F_{p^n}\to \mathbb F_{p^n}$ given by $h \mapsto xh$ is a linear map.
Let us write the matrix for $\phi$ in the basis above
\[
\phi(1, x,\ldots, x^{n-2}, x^{n-1}) = (1, x,\ldots, x^{n-2}, x^{n-1})
\begin{pmatrix}
{0}&{0}&{\ldots}&{0}&{-c_0}\\
{1}&{0}&{\ldots}&{0}&{-c_1}\\
{0}&{1}&{\ddots}&{0}&{-c_2}\\
{\vdots}&{\vdots}&{\ddots}&{\vdots}&{\vdots}\\
{0}&{0}&{\ldots}&{1}&{-c_{n-1}}\\
\end{pmatrix}
\]
We denote this matrix by $A$.
An element $h= a_0 + a_1 x + \ldots + a_{n-1}x^{n-1}$ is described by the vector $v = (a_0,a_1,\ldots,a_{n-1})^t$.
Then, the element $xh$ has coordinates described by the vector $Av$.
Hence, the random generator is described as follows.
We fix some vector $v = (a_0,a_1,\ldots,a_{n-1})^t$ and produce a sequence $v, Av, A^2v, A^3v, \ldots$.
The Fibonacci approach is similar but we use $A^t$ instead of $A$.
There is a way to describe Fibonacci approach in terms of the field $\mathbb F_{p^n}$ but I do not want to do this.

\subsection{Stream cipher}

There is an application of random generators in cryptography.
I want to discuss stream ciphers.
Usually, if you want to communicate, you have a secure channel and an open channel.
And it is more expensive to transfer data via the secure channel because of the computational overhead.
Hence, you want to minimize using the secure channel.
However, you still want to have some security transferring the data via the open channel.
You want it to be chip, fast, and secure.
Here is the method.

We fix a two symbol alphabet $\mathbb Z_2=\{0,1\}$.
Our message is a sequence $a_0, a_1, a_2,a_3,\ldots$.
Suppose, we have a random generator $S\colon \mathbb N \to \mathbb Z_2$, then we can generate a sequence $s_0 = S(0), s_1 = S(1), s_2 = S(2),\ldots$.
Then, we can encode the data replacing $a_k$  by $d_k = a_k + s_k\pmod 2$ and broadcast $d_k$ instead.
In order to decode the message, we compute $a_k = d_k + s_k \pmod 2$.
However, this means that the other person must know the whole sequence $s_k$.
Instead of transferring the sequence $s_k$ using a secure channel, we send a copy of the generator via the secure channel and then broadcast the information using above method via the open channel.

Let me demonstrate the basic transferring process using the following diagram
\[
\xymatrix{
	{}&{\boxed{\text{Random Generator}}}\ar[d]^{s_k}\ar@{<->}[rr]^{\text{syncronized}}&{}&{\boxed{\text{Random Generator}}}\ar[d]^{s_k}&{}\\
	{\boxed{\text{Sender}}}\ar[r]^{a_k}&{\boxed{+}}\ar[rr]^{d_k = a_k + s_k}&{}&{\boxed{+}}\ar[r]^{a_k = d_k + s_k}&{\boxed{\text{Receiver}}}\\
}
\]

This method can be absolutely insecure if all the components are not well chosen.
From the other hand, modern standards of GSM use some enhanced variants of this idea.
So, the appropriate use of this approach leads to very reliable results.
