\ProvidesFile{lecture10.tex}[Lecture 10]


\subsection{Proofs}

Now I have a lot of debts to pay.
The main one is to prove the Buchberger criterion.
The poof consists of two steps: the first one is to prove the Diamond Lemma and the second one is to derive the criterion from the lemma.

\begin{claim}
[The Diamond Lemma]
Suppose $F$ is a field, we are given a lexicographical order on monomials in $n$ variables, and $G\subseteq F[x_1,\ldots,x_n]\setminus\{0\}$ is any set.
Then, the following conditions are equivalent

\begin{tabular}{cc}
{
\parbox{12cm}{
\begin{enumerate}
\item $G$ is a Gr\"obner Basis.

\item for any polynomial $f\in F[x_1,\ldots,x_n]$ and any elementary reductions $f\stackrel{g_1}{\longrightarrow}f_1$ and $f\stackrel{g_2}{\longrightarrow}f_2$ (where $g_1,g_2\in G$) there is $f'\in F[x_1,\ldots,x_n]$ such that $f_1\stackrel{G}{\rightsquigarrow}f'$ and $f_2\stackrel{G}{\rightsquigarrow}f'$.%
\footnote{Here, we assume that there is at least one reduction of $f_1$ to $f'$ and at least one reduction of $f_2$ to $f'$.}
\end{enumerate}
}}&{
\parbox{4cm}{
\[
\xymatrix@R=15pt@C=15pt{
	{}&{\textcolor{red}{f}}\ar[dl]_{g_1}\ar[dr]^{g_2}&{}\\
	{\textcolor{OliveGreen}{f_1}}\ar[d]&{}&{\textcolor{OliveGreen}{f_2}}\ar[d]\\
	{}\ar@{}[d]|{\vdots}&{}&{}\ar@{}[d]|{\vdots}\\
	{}\ar[dr]&{}&{}\ar[dl]\\
	{}&{f'}&{}\\
}
\]
}
}
\end{tabular}
\end{claim}
\begin{proof}
(1)$\Rightarrow$(2).
Let $r_1$ be a remainder of $f_1$ with respect to $G$ and $r_2$ be a remainder of $f_2$ with respect to $G$.
In particular $f_1 \stackrel{G}{\rightsquigarrow} r_1$ and $f_2 \stackrel{G}{\rightsquigarrow} r_2$.
By definition $r_1$ and $r_2$ are also remainders of $f$.
Since $G$ is a Gr\"obner basis, all remainders of $f$ are the same.
Hence, $r_1 = r_2$.
Hence, $f' = r_1 = r_2$ satisfies the requirements.

(2)$\Rightarrow$(1).
We need to show that each polynomial $f\in F[x_1,\ldots,x_n]$ has only one remainder with respect to $G$.
Suppose the contrary holds and there is a polynomial $f$ with at least two different remainders.
Let us consider all possible elementary reductions of $f$ with respect to elements of $G$.
There are two possible cases:
\begin{enumerate}
\item All elementary reductions produce a polynomial with a unique remainder.

\item There is at least one reduction leading to a polynomial with several distinct remainders.
\end{enumerate}
I want to replace $f$ by a polynomial satisfying the first property.
If $f$ already satisfies the property, we are done.
Suppose that there is at least one reduction leading to $f_1$ such that $f_1$ has several distinct remainders.
If $f_1$ satisfies property (1), we replace $f$ by $f_1$.
Otherwise, there is an elementary reduction of $f_1$ leading to $f_2$ such that $f_2$ has several distinct remainders.
Hence, we get a sequence of reductions
\[
f\stackrel{g_1}{\longrightarrow}f_1\stackrel{g_2}{\longrightarrow}f_2\stackrel{g_3}{\longrightarrow}f_3\stackrel{g_4}{\longrightarrow}\ldots
\]
By Claim~\ref{claim::ReductFin}, the sequence is finite.
Hence, at some finite step, we get a polynomial $f_k$ such that it has several distinct remainders and satisfies property~(1) above.
We can demonstrate the process on the diagram below.
The red color means that the polynomial has several different remainders.
The green color means that the polynomial has a unique remainder.
\[
\xymatrix@R=15pt@C=15pt{
	{}&{}&{\textcolor{red}{f}}\ar[dll]\ar[d]\ar[dr]\ar[drrr]&{}&{}&{}\\
	{\textcolor{red}{*}}&{\ldots}&{\textcolor{red}{f_1}}\ar[dll]\ar[d]\ar[dr]\ar[drrr]&{\textcolor{OliveGreen}{*}}&{\ldots}&{\textcolor{OliveGreen}{*}}\\
	{\textcolor{red}{*}}&{\ldots}&{\textcolor{red}{f_2}}\ar[dll]\ar[d]\ar[dr]\ar[drrr]&{\textcolor{OliveGreen}{*}}&{\ldots}&{\textcolor{OliveGreen}{*}}\\
	{}&{\ldots}&{}\ar@{}[d]|(.3){\vdots}&{}&{\ldots}&{}\\
	{}&{}&{\textcolor{red}{f_k}}\ar[dll]\ar[d]\ar[dr]\ar[drrr]&{}&{}&{}\\
	{\textcolor{OliveGreen}{*}}&{\ldots}&{\textcolor{OliveGreen}{*}}&{\textcolor{OliveGreen}{*}}&{\ldots}&{\textcolor{OliveGreen}{*}}\\
}
\]
How, we may assume that $f$ has at least two different remainders with respect to $G$ but all elementary reductions of $f$ produce polynomials with a unique remainder.
Suppose that $f  \stackrel{G}{\rightsquigarrow}  r_1$ and $f  \stackrel{G}{\rightsquigarrow}  r_2$ are two different remainders.
By definition, there are $g_1, g_2\in G$ and polynomials $f_1, f_2\in F[x_1,\ldots,x_n]$ such that 
\[
f  \stackrel{g_1}{\longrightarrow}f_1  \stackrel{G}{\rightsquigarrow} r_1
\quad\text{and}\quad
f  \stackrel{g_2}{\longrightarrow}f_2  \stackrel{G}{\rightsquigarrow} r_2
\]
From the other hand, by the hypothesis of the claim, there is $f'$ such that $f_1$ and $f_2$ are reducible to $f'$ by $G$.
Let us reduce it further to a remainder $r'$.
The diagram below summarizes what has been done:
\[
\xymatrix@R=15pt@C=15pt{
	{}&{}&{\textcolor{red}{f}}\ar[dl]_{g_1}\ar[dr]^{g_2}&{}&{}\\
	{}&{\textcolor{OliveGreen}{f_1}}\ar[dl]\ar[d]&{}&{\textcolor{OliveGreen}{f_2}}\ar[d]\ar[dr]&{}\\
	{}\ar@{}[d]|{\vdots}&{}\ar@{}[d]|{\vdots}&{}&{}\ar@{}[d]|{\vdots}&{}\ar@{}[d]|{\vdots}\\
	{}\ar[d]&{}\ar[dr]&{}&{}\ar[dl]&{}\ar[d]\\
	{r_1}&{}&{r'}&{}&{r_2}\\
}
\]
By the choice of $f$, $r_1 \neq r_2$ are two different remainders.
From the other hand, polynomials $f_1$ and $f_2$ have their own unique remainder.
Hence, every remainder of $f_1$ is the same.
But $r_1$ and $r'$ are remainders of $f_1$.
Therefore, $r_1 = r'$.
By the same argument $r_2 = r'$.
Hence $r_1 = r' = r_2$, a contradiction.
This contradiction completes the proof.
\end{proof}

Now I need some simple auxiliary results.

\begin{claim}
\label{claim::ReductionProps}
Suppose $F$ is a field, we are given a polynomial ring $F[x_1,\ldots,x_n]$, and a lexicographical order is fixed.
\begin{enumerate}
\item If $f\stackrel{G}{\rightsquigarrow}f'$ and $m$ is a monomial, then $mf \stackrel{G}{\rightsquigarrow}mf'$.

\item If $f_1 - f_2 \stackrel{G}{\rightsquigarrow}0$, then $f_1$ and $f_2$ can be reduced to the same element with respect to $G$, that is, there is a polynomial $r$ such that $f_1\stackrel{G}{\rightsquigarrow}r$ and $f_2\stackrel{G}{\rightsquigarrow}r$.
\end{enumerate}
\end{claim}
\begin{proof}
1) This one is simple.
If we apply a sequence of elementary reductions to $f$, just multiply all the formulas by $m$ and we get the required reduction for $mf$.

2) Suppose $f_1 - f_2 \stackrel{g_1}{\longrightarrow}\varphi_1 \stackrel{g_2}{\longrightarrow}\ldots  \stackrel{g_k}{\longrightarrow}\varphi_k = 0$.
We will prove the result using induction on $k$.

In case $k = 0$.
We have $f_1 - f_2 = 0$, hence $f_1 = f_2$ and there is nothing to prove.
Let us assume that $k > 0$.
We consider the first step $f_1 - f_2 \stackrel{g}{\longrightarrow}\varphi$.
Suppose that a monomial $m$ is reduced in this reduction.
Let us isolate the term with $m$ in $f_1$ and $f_2$, hence we can write $f_1 = a m + h_1$ and $f_2 = bm + h_2$, where $a,b\in F$ and monomial $m$ does not appear in $h_1$ and $h_2$.
Suppose $g = c_g m_g + g_0$, where $c_g$ is the leading coefficient and $m_g$ is the leading monomial, and $m = t m_g$.
Then the reduction looks like this
\[
f_1 - f_2 \stackrel{g}{\longrightarrow}(a-b)m + h_1 -h_2 - \frac{a-b}{c_g}t g = \varphi
\]
Since monomial $m$ appears in $f_1$ and $f_2$, we may apply the reduction of monomial $m$ with respect to $g$ to each of these polynomials
\[
f_1 \stackrel{g}{\longrightarrow}f_1'=f_1 - \frac{a}{c_g}tg,\quad
f_2 \stackrel{g}{\longrightarrow}f_2'=f_2 - \frac{b}{c_g}tg
\]
It is clear that $\varphi = f_1' - f_2'$.
Hence, we may apply induction to $f_1'-f_2'$ and get the result.
\end{proof}

\begin{claim}
[The Buchberger criterion]
Suppose $F$ is a field, we are given a lexicographical order on monomials in $n$ variables, and $G\subseteq F[x_1,\ldots,x_n]\setminus\{0\}$ is any set.
Then, the following conditions are equivalent
\begin{enumerate}
\item $G$ is a Gr\"obner basis.

\item For every $g_1,g_2\in G$, $S_{g_1\,g_2}$ is reducible to zero with respect to $G$.%
\footnote{There is at least one way to reduce each S-polynomial to zero.}
\end{enumerate}
\end{claim}
\begin{proof}
(1)$\Rightarrow$(2).
Suppose $g_1,g_2\in G$, $g_1 = c_1 m_1 + g'_1$, $g_2 = c_2m_2 + g'_2$, and the least common multiple of $m_1$ and $m_2$ is $m = t_1 m_1 = t_2 m_2$.
Assume that $f = c_2 t_1 g_1$.
Then $f$ is reduced to zero by $g_1$.
From the other hand, $f$ is reduced to $S_{g_1\,g_2}$ by $g_2$.
Indeed,
\[
f\stackrel{g_2}{\longrightarrow} f' = c_2 t_1 g_1 - c_1 t_2 g_2 = S_{g_1\,g_2}
\]
Hence, the remainder of $f'$ with respect to $G$ coincides with the remainder of $f$.
But $f$ reduces to zero and $G$ is a Gr\"obner basis.
Therefore $f'$ also reduces to zero.


(2)$\Rightarrow$(1).
Now, we need to show that $G$ is a Gr\"obner basis.
In order to do that, we will check the second condition of the Diamond Lemma.
Suppose we have an arbitrary $f\in F[x_1,\ldots,x_n]$ and two reductions $f \stackrel{g_1}{\longrightarrow}f_1$ and $f\stackrel{g_2}{\longrightarrow}f_2$, where $g_1,g_2\in G$.
We need to reduce $f_1$ and $f_2$ to the same polynomial $f'$ with respect to $G$.
Suppose $g_1 = c_1 m_1 + g_1'$ and $g_2 = c_2 m_2 + g_2'$, where $m_1$ and $m_2$ are the leading monomials of $g_1$ and $g_2$ respectively.
Suppose the reductions are $f_1 = f - a_1 t_1 g_1$ and $f_1 = f - a_2 t_2 g_2$, where $a_1$ and $a_2$ are coefficients and $t_1$ and $t_2$ are monomials.
Claim~\ref{claim::ReductionProps} implies that it is enough to show that $f_1 - f_2$ reduces to zero with respect to $G$, that is $a_2t_2g_2 - a_1 t_1 g_1 \stackrel{G}{\rightsquigarrow}0$.

Let us consider the leading monomials of $t_1g_1$ and $t_2 g_2$.
There are two cases: either the monomials are the same or they are different.
Let us consider the second option first.

\paragraph{case 1}

We may assume that $M(t_2 g_2) > M(t_1 g_1)$.
Then the difference $a_2t_2g_2 - a_1 t_1 g_1$ has $M(t_2 g_2)$ as its leading monomial.
Hence, we may reduce this monomial with respect to $g_2$, that is
\[
a_2t_2g_2 - a_1 t_1 g_1\stackrel{g_2}{\longrightarrow} a_2t_2g_2 - a_1 t_1 g_1 - a_2t_2 g_2 = - a_1 t_1 g_1 \stackrel{g_1}{\longrightarrow} 0
\]

\paragraph{case 2}

Now we consider the case $m = M(t_1 g_1) = M(t_2 g_2)$.
Let us compute the leading coefficients of the polynomials $a_1t_1 g_1$ and $a_2 t_2 g_2$.
The coefficients are $a_1 c_1$ and $a_2 c_2$, respectively.
Now, there are two different subcases: either the leading coefficients are the same or they are different.

\paragraph{case 2a}

Suppose $a_2 c_2 = a_1 c_1$.
Then we have $\lambda = a_2 / c_1 = a_1 / c_2$.
Let us show that $a_2t_2g_2 - a_1 t_1 g_1$ is proportional to the S-polynomial of $g_1$ and $g_2$.
Let $d$ be the greatest common divisor of $t_1$ and $t_2$.
Then, $t_1 = d t_1'$ and $t_2 = d t_2'$.
Then,
\begin{gather*}
a_2t_2g_2 - a_1 t_1 g_1 = \lambda d \left(c_1 t_2' g_2 - c_2 t_1' g_1\right) = \lambda d S_{g_2\,g_1}
\end{gather*}
By the hypothesis of the claim $S_{g_2\, g_1}$ reduces to zero.
Then by Claim~\ref{claim::ReductionProps} item~(1), $\lambda d S_{g_1\,g_1}$ also reduces to zero.

\paragraph{case 2b}

Now we consider the case $a_2 c_2 \neq a_1 c_1$.
In this case the monomial $m = M(t_2 g_2) = M(t_1 g_1)$ does not vanish in the difference $a_2t_2g_2 - a_1 t_1 g_1$.
Let us reduce it with respect to $g_1$, that is
\[
a_2t_2g_2 - a_1 t_1 g_1 \stackrel{g_1}{\longrightarrow}
a_2t_2g_2 - a_1 t_1 g_1 - \frac{a_2 c_1 - a_1 c_1}{c_1}t_1 g_1 = a_2 t_2 g_2 - \frac{a_2 c_2}{c_1}t_1 g_1
\]
But now we are in the \textbf{case 2a} because $a_2 c_2 = \frac{a_2 c_2}{c_1}c_1$.
Hence the latter polynomial reduces to zero by the previous case and we are done.
\end{proof}

\begin{claim}
[The Dickson Lemma]
Suppose we are given a sequence of monomials on variables $x_1,\ldots,x_n$:
\[
m_1,m_2,\ldots, m_k,\ldots
\]
such that $m_i \mathbin{\not|} m_j$ whenever $i < j$.
Then the sequence must be finite.
\end{claim}
\begin{proof}
We will prove the claim by induction on the number of variables $n$.
If $n = 1$, then we have
\[
x_1^{d_1}, x_1^{d_2},\ldots, x_1^{d_k}
\]
Since the monomials on the left do not divide the monomials on the right, the sequence $d_k$ must be strictly decreasing and hence finite.

Now we would like to prove, that if the result holds for $n-1$ it is true for $n$.
Assume the claim is wrong and there is an infinite sequence with the given property.
Since $m_1$ does not divide $m_i$ there is a variable $x_{s_i}$ such that the degree of $m_i$ in $x_{s_i}$ is strictly smaller then the degree of $m_1$ in $x_{s_i}$.
Since the sequence of $x_{s_i}$ is infinite but the set of variables is finite, one of these variables appears infinitely many times.
Let us assume that this variable is $x_n$.

Hence, passing to a subsequence, we may assume that the degree of $m_k$ in $x_n$ is strictly smaller than that of $m_1$.
So, the sequence is of the form
\[
m_1(x_1,\ldots,x_{n-1})x_n^{d_1}, m_2(x_1,\ldots,x_{n-1})x_n^{d_2},\ldots, m_k(x_1,\ldots,x_{n-1})x_n^{d_k}, \ldots
\]
and $d_k < d_1$ for all $k > 1$.
Since the sequence $d_k$ is infinite and takes finitely many values (from $0$ to $d_1-1$), one of the values is achieved infinitely many times.

Hence, passing to a subsequence we may assume that the degree of $x_n$ is constant and equals $d$.
Thus, we may assume that the sequence is of the form
\[
m_1(x_1,\ldots,x_{n-1})x_n^{d}, m_2(x_1,\ldots,x_{n-1})x_n^{d},\ldots, m_k(x_1,\ldots,x_{n-1})x_n^{d}, \ldots
\]
But since the degree of $x_n$ is the same, the condition $m_i\mathbin{\not |} m_j$ implies that $m_i(x_1,\ldots,x_{n-1})\mathbin{\not |} m_j(x_1,\ldots,x_{n-1})$.
Hence, we have an infinite sequence of monomials depending on $n-1$ variable such that $m_i\mathbin{\not |} m_j$ whenever $i < j$.
This contradiction completes the proof.
\end{proof}

\begin{claim}
The Buchberger algorithm terminates.
\end{claim}
\begin{proof}
Let me recall the algorithm.
We are given $F\subseteq F[x_1,\ldots, x_n]$ and a lexicographical ordering is fixed.
We do the following steps:
\begin{enumerate}
\item $G = F$.

\item For all $f, g\in F$ we compute $S_{f\,g}\stackrel{G}{\rightsquigarrow} r_{f\, g}$.
If all the remainders are zero, the algorithm terminates.

\item We set $G = G \cup \{r_{f\,g}\mid r_{f\,g}\neq 0\}$ and repeat from Step $2$.
\end{enumerate}
Let us assume that the algorithm never stops.
Then on each step we add at least one polynomial $r_{f\, g}$.
Suppose we added $r_1$ on the first loop, then we added $r_2$, then $r_3$, and so on.
Then $r_k$ is reduced by all the polynomials $r_1,\ldots, r_{k-1}$.
In particular, the leading monomial $M(r_k)$ is not divisible by the leading monomials $M(r_1), \ldots, M(r_{k-1})$.
Hence, the sequence $M(r_1), M(r_2),\ldots, M(r_k),\ldots$ contradicts the Dickson Lemma.
This contradiction completes the proof.
\end{proof}