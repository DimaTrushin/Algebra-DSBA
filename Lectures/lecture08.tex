\ProvidesFile{lecture08.tex}[Lecture 8]


\section{Gr\"obner bases}

Now I want to discuss polynomials in several variables.
One of the main reason why we love polynomials in one variable is the Euclidean division algorithm.
It turns out that we can solve almost any problem effectively using this simple division algorithm.
However, if we switch to polynomials in several variables there is no Euclidean division algorithm (one can even prove this rigorously).
Instead of rubbing our tearing eyes regretting about switching to several variables we will follow a better plan.
We can find a procedure very similar to Euclidean division algorithm, it is called reduction.
Moreover, we will be able to reduce a polynomial with respect to a family of polynomials, that is, we will divide one polynomials by a family of polynomials with remainder.
There will be one problem.
The division process is not unique and we may get different remainders sometimes.
At this point we call Gro\"ebner basis to the rescue.
A Gr\"obner basis will provide us with a good reduction algorithm and ensure the uniqueness of all remainders.

\subsection{Polynomials in several variables}

Let $F$ be a field.
A polynomial $f$ in variables $x_1,\ldots,x_n$ is a picture
\[
f = \sum_{k_1,\ldots,k_n\geqslant 0} a_{k_1\ldots k_n} x_1^{k_1} \ldots x_n^{k_n},\quad a_{k_1\ldots k_n} \in F
\]
where only finitely many coefficients are nonzero.
The addition and multiplication are given by the usual rules.
The set of all polynomials in variables $x_1,\ldots,x_n$ with coefficients in $F$ will be denoted by $F[x_1,\ldots,x_n]$.
The set $F[x_1,\ldots,x_n]$ with addition and multiplication is a commutative ring.

The expression $m = x_1^{k_1}\ldots x_n^{k_n}$ is called monomial.
The degree of the monomial is $\deg m = k_1+ \ldots + k_n$.
The degree of a polynomial $f$ is the maximum of the degrees of all monomials appearing in $f$ with nonzero coefficient.
The degree of the zero polynomial is $-\infty$ as in the case of one variable.
A straightforward computation shows that, if $f,g\in F[x_1,\ldots,x_n]$, then $\deg(fg) = \deg f + \deg g$.
A monomial multiplied by an element of the field $F$ is called a term.
Hence, a polynomial is always a sum of terms.

\subsection{Monomial orderings}

The first ingredient for the general division algorithm is the ordering on monomials.
We need to know when a monomial is less than another one.
There are many different orderings.
However, I am going to simplify your life and will stick to lexicographical order only.

\begin{definition}
We want to define a lexicographical order on monomials.
\begin{enumerate}
\item We need to fix an ordering on the variables $x_1,\ldots,x_n$.
For example $x_1 > x_2 > \ldots > x_n$.
However, we can take any permutation of the variables.

\item Suppose we fixed the ordering $x_1 > \ldots > x_n$ on the variables.
Now, we are ready to define the corresponding lexicographical order $\Lex(x_1,\ldots,x_n)$ on the monomials.
\end{enumerate}
Let $m = x_1^{k_1}\ldots x_n^{k_n}$ and $m'= x_1^{k_1'}\ldots x_n^{k_n'}$ be two monomials.
Then we compare $k_1$ and $k_1'$.
If $k_1 > k_1$, then $m > m'$.
If $k_1 < k_1'$, then $m < m'$.
If $k_1 = k_1'$, then we compare $k_2$ and $k_2'$ and repeat the algorithm above.
In particular, $m > m'$ if and only if there exists $1\leqslant j \leqslant n$ such that $k_1 = k_1',\ldots, k_{j-1} = k_{j-1}'$ and $k_j > k_j'$.
\end{definition}

\begin{examples}
Consider the ring $F[x, y, z]$ and monomials $m_1 = x^2 y z^4$, $m_2 = x y^3 z$, and $m_3 = x y z^5$.
There are $6$ ways to order the variables $x, y, z$.
In each case, we define the corresponding lexicographical order and compare the monomials.
\begin{enumerate}
\item If $x > y > z$, then $m_1 > m_2 > m_3$.

\item If $x > z > y$, then $m_1 > m_3 > m_2$.

\item If $y > x > z$, then $m_2 > m_1 > m_3$.

\item If $y > z > x$, then $m_2 > m_3 > m_1$.

\item If $z > x > y$, then $m_3 > m_1 > m_2$.

\item If $z > y > x$, then $m_3 > m_1 > m_2$.
\end{enumerate}
\end{examples}


\begin{remarks}
Here I want to list several properties of any lexicographical order.
All monomials below depend on $n$ variables and we are given a lexicographical order.
\begin{enumerate}
\item For any monomial $m\neq 1$, we have $1 < m$.

\item If $m$ and $m'$ are monomials such that $m'$ divides $m$, that is, $m = m' t$ for some other monomial $t$, then $m' \leqslant m$ and if $t\neq 1$, the inequality is strict.

\item if $m, t, s$ are monomials such that $m \leqslant t$, then $ms \leqslant ts$ and if $m < t$, then $ms < ts$.

\end{enumerate}
\end{remarks}



\begin{claim}
\label{claim::LexWellOrder}
Suppose we fixed a lexicographical order on monomials in $n$ variables and $m_1 > m_2 > \ldots > m_k > \ldots$ is a strictly descending chain of monomials in variables $x_1,\ldots,x_n$.
Then the chain is finite.
\end{claim}
\begin{proof}
First of all, we may rename the variables $x_1,\ldots,x_n$ such that they go in the decreasing order $x_1>x_2>\ldots>x_n$.
Now, suppose the contrary holds and there is an infinite sequence $m_1>m_2>\ldots>m_k>\ldots$.
I will write the monomials explicitly on the following diagram.
\[
\text{First}\quad
\xymatrix@R=10pt@C=10pt{
	{x_1^{k_1(1)}}&{x_2^{k_2(1)}}&{\ldots}&{x_n^{k_n(1)}}\\
	{x_1^{k_1(2)}}\ar@{}[u]|{\rotatebox{90}{$\leqslant$}}&{x_2^{k_2(2)}}&{\ldots}&{x_n^{k_n(2)}}\\
	{\vdots}\ar@{}[u]|{\rotatebox{90}{$\leqslant$}}&{\vdots}&{}&{\vdots}\\
	{x_1^{k_1(r_1)}}\ar@{}[u]|{\rotatebox{90}{$\leqslant$}}&{x_2^{k_2(r_1)}}&{\ldots}&{x_n^{k_n(r_1)}}\\
	{x_1^{k_1(r_1+1)}}\ar@{=}[u]&{x_2^{k_2(r_1+1)}}&{\ldots}&{x_n^{k_n(r_1+1)}}\\
}
\quad\text{Second}\quad
\xymatrix@R=10pt@C=10pt{
	{x_1^{k_1}}&{x_2^{k_2(r_1)}}&{\ldots}&{x_n^{k_n(r_1)}}\\
	{x_1^{k_1}}\ar@{=}[u]&{x_2^{k_2(r_1+1)}}\ar@{}[u]|{\rotatebox{90}{$\leqslant$}}&{\ldots}&{x_n^{k_n(r_1+1)}}\\
	{\vdots}\ar@{=}[u]&{\vdots}\ar@{}[u]|{\rotatebox{90}{$\leqslant$}}&{}&{\vdots}\\
	{x_1^{k_1}}\ar@{=}[u]&{x_2^{k_2(r_2)}}\ar@{}[u]|{\rotatebox{90}{$\leqslant$}}&{\ldots}&{x_n^{k_n(r_2)}}\\
	{x_1^{k_1}}\ar@{=}[u]&{x_2^{k_2(r_2+1)}}\ar@{=}[u]&{\ldots}&{x_n^{k_n(r_2+1)}}\\
}
\]
First, we look at the degree of $x_1$.
By the definition of the lexicographical order the degrees must go in the descending order, that is, $k_1(1) \geqslant k_1(2)\geqslant \ldots$.
However, this is a sequence of natural numbers.
Hence, must stabilize in the sense there is a number $r_1$ such that $k_1(r_1) = k_1(r_1+1) = \ldots$.
We denote this number by $k_1$.
Hence, for all monomials $m_i$ with $i \geqslant r_1$, the degree of $x_1$ is $k_1$.

Now, we look at the degree of $x_2$ and the sequence of monomials $m_{r_1}>m_{r_1+1}>\ldots$.
Since the degree of $x_1$ is the same for all monomials, we must have $k_2(r_1)\geqslant k_2(r_1+1)\geqslant \ldots$.
Again, this is a descending chain of natural numbers.
Hence, there is a number $r_2 > r_1$ such that $k_2(r_2) = k_2(r_2+1) = \ldots$.
We denote this number by $k_2$.
Hence, for all monomials $m_i$ with $i \geqslant r_2$, the  degrees of $x_1$ and $x_2$ are  $k_1$ and $k_2$, respectively.

We may proceed in the similar way and find $r_3 > r_2$ such that all degrees of $x_3$ are the same for $m_i$ whenever $i\geqslant r_3$.
Then we find $r_4 > r_3$ such that all degrees of $x_4$ are the same for $m_i$ whenever $i \geqslant r_4$ and so on.
Finally, we find $r_n > r_i$ such that all the degrees of all variables are the same for the monomials $m_i$ such that $i \geqslant r_n$.
But then all such monomials are equal.
In particular, the comparison $m_{r_n}>m_{r_n+1}$ is impossible.
This contradiction completes the proof.
\end{proof}


\begin{remark}
There is a common misunderstanding here.
There is no infinite strictly descending chain of monomials.
However, for a given monomial, there are infinitely many strictly smaller ones as the following example shows.
Let us consider $\mathbb Q[x, y]$ with the order $x > y$ and the corresponding lexicographical order.
Then there are infinitely many monomials strictly smaller than $x^2$.
Indeed, all monomials of the form $xy^n$ are strictly smaller than $x^2$.
Hence, we may produce a sequence
\[
x^2 > xy^n > xy^{n-1} > \ldots > xy > x > y^m > y^{m-1} > \ldots > y > 1
\]
Each descending chain of monomials starting at $x^2$ is finite but the length of the chain can be arbitrary long.
\end{remark}


\subsection{Reduction}

Since we know how to compare the monomials, we are ready to define the reduction algorithm.
First we need to define an elementary reduction and then an arbitrary reduction.
Let me recall the context.
We are given a field $F$, a polynomial ring $F[x_1,\ldots,x_n]$, and a lexicographic ordering on the monomials in $n$ variables.


\begin{definition}
Suppose $F$ is a field, we fix a lexicographical order on the monomials of $F[x_1,\ldots,x_n]$, and $f\in F[x_1,\ldots,x_n]$ is an arbitrary nonzero polynomial.
Then, the polynomial $f$ can be written as
\[
f = c_1 m_1 + c_2 m_2 + \ldots + c_k m_k,\quad c_i\in F,\;m_i\;\text{are monomials such that}\; m_1 > m_2 >\ldots>m_k
\]
The term $c_1m_1$ is called the leading term and will be denoted by $T(f)$.
The monomial $m_1$ is called the leading monomial of $f$ and will be denoted by $M(f)$.
The coefficient $c_1$ is called the leading coefficient of $f$ and is denoted by $C(f)$.
It is clear by definition that $T(f) = C(f) M(f)$.
The part $c_2m_2 + \ldots + c_k m_k$ is called the tail of $f$ and will be denoted by $f_0$.
Hence, $f$ can be written as $f = T(f) + f_0$.
\end{definition}


\begin{definition}
Suppose $g\in F[x_1,\ldots,x_n]$ is a nonzero polynomial and $f\in F[x_1,\ldots,x_n]$ is any polynomial.
Assume that
\[
f= c_1m_1 + \ldots + c_i m_i + \ldots + c_k m_k,\quad c_i\in F,\;m_i\;\text{are monomials such that}\; m_1 > m_2 >\ldots>m_k
\]
and
\[
g = C(g) M(g) + g_0 = T(g) + g_0
\]
We take $m$ to be $m_i$, that is a monomial in $f$ and assume that $m$ is divisible by the leading monomial of $g$, that is $m = t M(g)$.
We define an elementary reduction of $f$ with respect to $g$ as 
\[
f\stackrel{g}{\longrightarrow} f' = f - \frac{c_i}{C(g)}t g
\]
The polynomial $f'$ is the result of the elementary reduction.
\end{definition}

In short, the elementary reduction works as follows: we find a monomial  $m_i$ of $f$ divisible by $M(g)$ and replace it by the tale of $g$ multiplied by $- c_i m_i / T(g)$.

\begin{example}
Let us consider the ring $\mathbb Q[x, y, z]$, $f = xyz$, $g_1 = xy - z$, and $g_2 = yz - 1$.
Then
\[
f\stackrel{g_1}{\longrightarrow} xyz - z(xy - z) = z^2,
\quad\text{or}\quad
f\stackrel{g_2}{\longrightarrow} xyz - x(yz - 1) = x
\]
\end{example}

\begin{definition}
Suppose $G\subseteq F[x_1,\ldots,x_n]\setminus \{0\}$ is a set of polynomials and $f, f'\in F[x_1,\ldots,x_n]$ are any polynomials.
We say that $f$ is reducible to $f'$ with respect to $G$ if there is a finite sequence of elementary reductions as below%
\footnote{The polynomials $g_1,\ldots,g_k$ need not be distinct.}
\[
f\stackrel{g_1}{\longrightarrow}f_1\stackrel{g_2}{\longrightarrow}f_2\stackrel{g_3}{\longrightarrow}\ldots \stackrel{g_k}{\longrightarrow}f_k = f'\quad\text{where}\;g_i \in G
\]
In this case we will write $f\stackrel{G}{\rightsquigarrow} f'$.

If the polynomial $f'$ is not reducible by any $g\in G$, we say that $f'$ is a remainder of $f$ with respect to $G$.
\end{definition}


\begin{remarks}
\begin{enumerate}
\item A polynomial $f'$ is not reducible by any $g\in G$ if and only if every monomial of $f'$ is not divisible by $M(g)$ for any $g\in G$.

\item It should be noted that there could be different remainders of a polynomial $f$ with respect to a set $G$.
Here is an example.
We deal with the ring $\mathbb Q[x, y, z]$ and the lexicographical order is given by $x > y > z$.
Suppose $f = xyz$, $g_1 = xy - 1$, $g_2 = yz - 1$, and $G = \{g_1, g_2\}$.
Then,
\[
f\stackrel{g_1}{\longrightarrow}xyz - z(xy - 1) = z
\quad\text{and}\quad
f\stackrel{g_2}{\longrightarrow}xyz - x(yz - 1) = x
\]
Then by definition $f\stackrel{G}{\rightsquigarrow} z$ and $f\stackrel{G}{\rightsquigarrow} x$.
Moreover, the polynomials $z$ and $x$ are not reducible with respect to $G$.
Hence, they are different remainders of $f$ with respect to $G$.
\end{enumerate}
\end{remarks}

Hence, in general a remainder of a polynomial $f$ with respect to some set of polynomials $G$ is not unique.
The latter happens because the set $G$ was not good enough.
This leads us to the notion of Gr\"obner basis.

\begin{definition}
[Gr\"obner basis]
Suppose $F$ is a field, $G\subseteq F[x_1,\ldots,x_n]\setminus\{0\}$, and we fix a lexicographical order on the monomials.%
\footnote{It should be noted the the notion of a Gr\"obner basis highly depends on the choice of the order.
If we change the ordering on the variables, the set $G$ may become a Gr\"obner basis even if it was not one with respect to the old ordering and vice verse.}
We say that $G$ is a Gr\"obner basis if for every $f\in F[x_1,\ldots,x_n]$ all its remainders are the same.
\end{definition}

There is a hidden problem here.
Suppose we are given a polynomial $f$ and a set of polynomials $G$.
If $f$ is reducible with respect to $G$ we can make an elementary reduction and get $f_1$.
Then if $f_1$ is reducible, we can make another elementary reduction and get $f_2$ and so on.
A natural question is: does the process stop?
The answer to this question is yes as the Claim~\ref{claim::ReductFin} shows.

\begin{examples}
We will see a lot of examples of Gr\"obner bases latter.
Now, I want to give some simple examples without explanations because we are not ready to prove that the examples below are Gr\"obner bases.
\begin{enumerate}
\item For any polynomial ring $F[x_1,\ldots,x_n]$ with any lexicographical order the set $G = \{g\}$ consisting of one nonzero polynomial is always a Gr\"obner basis.

\item Suppose $F[x,y,z,w]$ and we are given any lexicographical order, then the set $G = \{xy - 1, zw + 1\}$ is a Gr\"obner basis.
This happens because the leading monomials of the polynomials of $G$ are coprime.
\end{enumerate}
\end{examples}


\begin{definition}
Suppose $F$ is a field and we fix some lexicographical order on the monomials in $n$ variables.
As before, each polynomial $f\in F[x_1,\ldots,x_n]$ can be written as
\[
f = c_1 m_1 + c_2 m_2 + \ldots + c_k m_k,\quad c_i\in F,\;m_i\;\text{are monomials such that}\; m_1 > m_2 >\ldots>m_k
\]
We denote its $i$-th largest monomial $m_i$ by $M_i(f)$.
In particular, $M_1(f)$ is the leading monomial of $f$, $M_2(f)$ is the next largest monomial of $f$ etc.
The $i$-th largest monomial need not exist if $f$ contains less than $i$ monomials.
\end{definition}



\begin{claim}
\label{claim::MkReduct}
Suppose $F$ is a field, we are given $F[x_1,\ldots,x_n]$, and some lexicographical order is fixed.
Suppose $f, f',g\in F[x_1,\ldots,x_n]$ are such that $f\stackrel{g}{\longrightarrow}f'$ and $M_1(f) = M_1(f'), \ldots, M_{k-1}(f) = M_{k-1}(f')$, and monomials $M_k(f)$ and $M_k(f')$ exist.
Then, $M_k(f) \geqslant M_k(f')$.
\end{claim}
\begin{proof}
Suppose $f = c_1 m_1 + \ldots + c_{k-1} m_{k-1} + c_km_k +\ldots + c_s m_s$, where $c_i\in F$ and $m_i$ are monomials written in decreasing order with respect to the lexicographical order.
The polynomial $f'$ is obtained from $f$ via one elementary reduction.
The latter means there is a monomial $m_i$ in $f$ divisible by $M(g)$ and we replace $m_i$ by a linear combination of smaller monomials.
By the hypothesis the first $k-1$ monomials of $f$ and $f'$ are the same.
Hence, $i\geqslant k$.
If $i>k$ then, the $k$-th monomial of $f'$ is the same as in $f$.
This means that $M_k(f) = M_k(f')$.
If $i = k$, then we replaced $m_k$ by a linear combination of smaller monomials.
For example the result may look like this:
\[
\xymatrix@R=8pt@C=10pt{
	{f}\ar@{<->}[rr]&{}&{m_1}&{\ldots}&{m_{k-1}}&{m_k}&{}&{m_{k+1}}&{}&{\ldots}&{m_s}&{}\\
	{f'}\ar@{<->}[rr]&{}&{m_1}&{\ldots}&{m_{k-1}}&{}&{m'_k}&{m'_{k+1}}&{m'_{k_2}}&{\ldots}&{}&{m'_{s'}}\\
}
\]
But then, the monomial $m'_k = M_k(f')$ and is strictly smaller than $m_k = M_k(f)$ and we are done.
\end{proof}

\begin{claim}
\label{claim::ReductFin}
Suppose $F$ is a field, we are given $F[x_1,\ldots,x_n]$, and some lexicographical order is fixed.
Then, for every polynomial $f\in F[x_1,\ldots,x_n]$ and any subset $G\subseteq F[x_1,\ldots,x_n]\setminus\{0\}$, any sequence of elementary reductions of $f$ with respect to $G$ is finite.
\end{claim}
\begin{proof}
Suppose the contrary holds and there is an infinite sequence of elementary reductions
\[
f\stackrel{g_1}{\longrightarrow}f_1\stackrel{g_2}{\longrightarrow}f_2\stackrel{g_3}{\longrightarrow}\ldots \stackrel{g_k}{\longrightarrow}f_k\stackrel{g_{k+1}}{\longrightarrow}\ldots
\]
Let us consider the sequence of leading monomials $M_1(f),M_1(f_1), M_1(f_2),\ldots$.
Applying Claim~\ref{claim::MkReduct} in case $k = 1$, we see that $M_1(f) \geqslant M_1(f_1)\geqslant M_1(f_2) \geqslant \ldots$.
By Claim~\ref{claim::LexWellOrder}, the sequence cannot be strictly decreasing and there is $r_1$ such that $M_1(f_{r_1}) = M_1(f_{r_1+1})=\ldots$.
We denote this common monomial by $m_1$.
Since the sequence of reductions is infinite, there must exist the second monomial in all polynomials $f_i$.
By Claim~\ref{claim::MkReduct} in case $k = 2$, we have $m_1 > M_2(f_{r_1})\geqslant M_2(f_{r_1 + 1})\geqslant \ldots$.
This descending chain of monomials cannot be strictly decreasing by Claim~\ref{claim::LexWellOrder}.
Hence, there is $r_2 > r_1$ such that $m_1 > M_2(f_{r_2}) = M_2(f_{r_2+1})=\ldots$.
We denote this common monomial by $m_2$ and get a sequence $m_1 > m_2$.

Now we repeat the arguments and consider the third largest monomials, then find $m_3$ such that $m_1 > m_2 > m_3$.
Proceeding in such a way we construct an infinite decreasing sequence of monomials $m_1 > m_2 > m_3 > \ldots > m_s> \ldots$.
The latter contradicts Claim~\ref{claim::LexWellOrder}.
This contradiction completes the proof.
\end{proof}

