\ProvidesFile{lecture05.tex}[Lecture 5]


\section{Rings and Fields}

\subsection{Definitions}

As time passes, we are no longer satisfied with an algebraic structure having one operation only.
We want more.
We are now mature enough to deal with two operations simultaneously.
So, let us go straight into the abyss.

\begin{definition}
We are going to define a ring $(R, +, \cdot)$ or simply $R$.
\begin{itemize}
\item\textbf{Data:} 
\begin{enumerate}
\item A set $R$ of elements.

\item An operation $+ \colon R\times R\to R$ called addition.

\item An operation $\cdot \colon R\times R\to R$ called multiplication.
\end{enumerate}

\item\textbf{Axioms:}
\begin{enumerate}
\item $(R, +)$ is an abelian group.

\item Multiplication is left and right distributive over addition:
\[
a(b + c) = ab + ac \quad\text{and}\quad (a + b) c = ac + bc\quad \text{for all  }a, b, c\in R
\]

\item Multiplication is associative: $(ab)c = a(bc)$ for all $a, b, c\in R$.

\item Multiplication has a neutral element denoted by $1$.
\end{enumerate}
\end{itemize}
In this case, we say that $(R, +, \cdot)$ is an associative ring with an identity.
We will use the term ring to denote an associative ring with an identity.
As before, we usually say that $R$ is a ring assuming that the operations in use are clear.
The neutral element with respect to addition is denoted by $0$ and is called zero.
If $a\in R$ is an arbitrary element, its inverse with respect to addition is denoted by $-a$.
For any $a, b\in R$, the expression $a + (-b)$ will be denoted by $a - b$ for short.
If in addition we have
\begin{itemize}
\item[]
\begin{enumerate}
\setcounter{enumi}{4}
\item Multiplication is commutative: $ab = ba$ for all $a, b\in R$.
\end{enumerate}
\end{itemize}
The ring is said to be commutative.
And if we add the following conditions
\begin{itemize}
\item[]
\begin{enumerate}
\setcounter{enumi}{5}
\item Every non-zero element is invertible with respect to multiplication: for every $a\in R\setminus\{0\}$, there exists an element $b\in R$ such that $ab = ba = 1$.

\item $1 \neq 0$.
\end{enumerate}
\end{itemize}
The ring is said to be a field.
In this case, the inverse element for $a$ is denoted by $a^{-1}$.%
\footnote{It should be noted that it is not enough to check one of the conditions $ab = 1$ or $ba = 1$ if the multiplication is not commutative.}
\end{definition}

\begin{examples}
\begin{enumerate}
\item Let $R = \{*\}$ be the set with one element only.
Then there is only one possible binary operation on this set.
We use this operation as addition and multiplication, that is $+\colon R\times R\to R$ is given by $* + * = *$ and $\cdot \colon R\times R\to R$ is given by $* \cdot * = *$.
Then this is a commutative ring.
Its only element $*$ is the zero element (neutral with respect to addition) and the identity element (neutral with respect to multiplication).
Every nonzero element is invertible because there is no nonzero elements, hence condition~(6) holds.
But $1 = 0$ in this ring.

\item The integer numbers with the usual addition and multiplication $(\mathbb Z, +, \cdot)$ is a commutative ring.

\item The ring of matrices with the usual matrix addition and multiplication $(\operatorname{M}_n(\mathbb R), +, \cdot)$ is a ring.

\item The real numbers with the usual addition and multiplication $(\mathbb R, +, \cdot)$ is a field.

\item The set of remainders modulo natural number $n$ with the usual addition and multiplication modulo $n$, that is $(\mathbb Z_n, +, \cdot)$, is a commutative ring.

\item If in the previous example the modulus $p$ is prime, then $(\mathbb Z_p, +,\cdot)$ is a field.% (see Claim~\ref{claim::ZpField}).

\item Let $A$ be any commutative ring and $x$ be a variable.
Then, $A[x] = \{a_0 + a_1 x + \ldots + a_n x^n \mid n\in \mathbb N,\; a_i\in A\}$ is the set of all polynomial with coefficients in $A$.
The set $A[x]$ with usual addition and multiplication becomes a commutative ring.
\end{enumerate}
\end{examples}

\begin{remark}
If we are given a ring $R$, then there are some natural properties of operations that are not included in the list of axioms but easily follow from them.
I do not want to torture you proving these formal tricks but it makes sense to list the useful properties:
\begin{enumerate}
\item For any element $x\in R$, we have $0 x = x 0  = 0$.

\item For any elements $x, y\in R$, we have $x - (-y) = x + y$.

\item For any element $x\in R$, we have $(-1) x = - x$.

\item For any invertible elements $x,y\in R$, $(xy)^{-1} = y^{-1}x^{-1}$.
\end{enumerate}
\end{remark}

\begin{definition}
Let $R$ be a ring.
We are going to define a subring $T\subseteq R$.
\begin{itemize}
\item\textbf{Data:} 
\begin{enumerate}
\item A subset $T\subseteq R$.
\end{enumerate}

\item\textbf{Axioms:}
\begin{enumerate}
\item $(T, +)\subseteq (R, +)$ is a subgroup.

\item $T$ is closed under multiplication.

\item $T$ contains $1$.
\end{enumerate}
\end{itemize}
\end{definition}

\begin{examples}
\begin{enumerate}
\item Consider $\mathbb Z\subseteq \mathbb R$.
The set $\mathbb Z$ is a subring.

\item Upper triangular matrices is a subring of all square matrices.

\item Scalar matrices is a subring of all square matrices.
\end{enumerate}
\end{examples}

\subsection{Elements of a ring}

There are many approaches to study a ring.
The simplest one is the element-wise approach.
In this case, we study elements with different properties.

\begin{definition}
Let $R$ be a ring and $x\in R$ be an element of $R$.
\begin{itemize}
\item The element $x$ is called invertible if there exists $y\in R$ such that $xy = yx = 1$.
In this case $y$ is denoted by $x^{-1}$.
The set of all invertible elements of $R$ is denoted by $R^*$.

\item The element $x$ is called left zero divisor if there exists a nonzero $y\in R$ such that $xy = 0$.
Similarly, $x$ is called right zero divisor if there exists a nonzero $y\in R$ such that $yx = 0$.
The sets of left and right zero divisors will be denoted by $D_l(R)$ and $D_r(R)$, respectively.
The set $D(R) = D_l(R) \cup D_r(R)$ is the set all all zero divisors of $R$.

\item The element $x$ is called nilpotent if $x^n = 0$ for some $n\in \mathbb N$.
The set of all nilpotent elements is denoted by $\operatorname{nil}(R)$.

\item The element $x$ is called idempotent if $x^2 = x$.
The set of all idempotents of $R$ is denoted by $E(R)$.

\end{itemize}
\end{definition}

\begin{examples}
\begin{enumerate}
\item Let $R = \mathbb Z$ be the ring of integers.
Then, $\mathbb Z^* = \{\pm 1\}$, $D(\mathbb Z) = 0$, $\operatorname{nil}(\mathbb Z) =  0$, $E(\mathbb Z) = \{1, 0\}$.

\item Let $R = \mathbb R$ be the field of real numbers.
Then, $\mathbb R^* = \mathbb R\setminus\{0\}$, $D(\mathbb R) = 0$, $\operatorname{nil}(\mathbb R) = 0$, $E(\mathbb R) = \{1, 0\}$.

\item Let $R = \operatorname{M}_n(\mathbb R)$ be the ring of square matrices.
Then, $\operatorname{M}_n(\mathbb R)^* = \operatorname{GL}_n(\mathbb R)$, $D(\operatorname{M}_n(\mathbb R))$ is the set of degenerate matrices, $\operatorname{nil}(\operatorname{M}_n(\mathbb R)$ is the set of matrices with zero complex eigenvalues, $E(\operatorname{M}_n(\mathbb R))$ is the set of matrices of the form $C^{-1}DC$, where $D$ is diagonal with elements $1$ and $0$ on the diagonal.

\item Let $R = \mathbb Z_n$ and $n = p_1^{k_1}\ldots p_r^{k_r}$, where $p_i$ are distinct prime numbers and $k_i > 0$.
Then, $\mathbb Z_n^* = \{k\in \mathbb Z_n\mid (k, n) = 1\}$, $D(\mathbb Z_n) = \{k\in \mathbb Z_n\mid (k, n) \neq 1\}$, $\operatorname{nil}(\mathbb Z_n) = \{k\in \mathbb Z_n\mid p_1|k,\ldots,p_r|k\}$.
By the Chinese Remainder Theorem, there exist elements $e_i\in \mathbb Z_n$ such that $e_i = 1 \pmod{p_i^{k_i}}$ and $e_i = 0\pmod{p_j^{k_j}}$ if $j\neq i$.
Then, $E(\mathbb Z_n)$ consists of the sums $\sum_{t} e_{i_t}$.
The empty sum denotes the zero and the sum through all the elements $e_i$ gives the identity.
\end{enumerate}
\end{examples}

\subsection{Ideals}

Another approach to study a ring is to study its special subsets.
It turns out that this approach is more convenient than the elements-wise one.
There are two interesting types of subsets: subrings and ideals.
The word ideal sounds much cooler, so I am going to deal with the ideals right now.

\begin{definition}
Suppose that $(R, +, \cdot)$ is a ring.
I am going to define an ideal $I$ in the ring $R$.
\begin{itemize}
\item\textbf{Data:} 
\begin{enumerate}
\item A subset $I\subseteq R$.
\end{enumerate}

\item\textbf{Axioms:}
\begin{enumerate}
\item $(I, +)\subseteq (R,+)$ is a subgroup.

\item For any $r\in R$ we have
\[
r I = \{rx\mid x\in I\} \subseteq I\quad\text{and}\quad Ir = \{xr\mid x\in I\}\subseteq I
\]
\end{enumerate}
\end{itemize}
In this case, we say that $I$ is an ideal of $R$.
The subsets $0$ and $R$ are always ideals and are called the trivial ideals of $R$.
\end{definition}

It should be noted that it is not enough to check that $rI \subseteq I$ for any $r$ or that $Ir \subseteq I$ for any $r$.
If the ring $R$ is not commutative, it may happen that on of the conditions holds while the other one does not.


\begin{claim}
\label{claim::ZIdeals}
Let $R = \mathbb Z$, then every ideal is of the form $n\mathbb Z$ for some $n\in \mathbb Z$.
\end{claim}
\begin{proof}
Let $I\subseteq \mathbb Z$ be an ideal.
Then $(I, +)$ is at least a subgroup in $(\mathbb Z, +)$.
By Claim~\ref{claim::Zsubgroups}, we already know that $I = n\mathbb Z$ for some $n\in \mathbb Z$.
From the other hand, let us take an arbitrary $I = n\mathbb Z$ and $k\in \mathbb Z$.
Then,
\[
k I = \{k x \mid x\in n\mathbb Z\} = \{kn m\mid m\in \mathbb Z\} = kn\mathbb Z\subseteq n\mathbb Z
\]
Hence, every additive subgroup is an ideal.
\end{proof}


\begin{claim}
\label{claim::ZnIdeals}
Let $R = \mathbb Z_n$, then every ideal is uniquely presented in the form $k\mathbb Z_n$ for some $k|n$.
\end{claim}
\begin{proof}

First, we should show that the set $k \mathbb Z_n$ is an ideal.
By Claim~\ref{claim::Znsubgroups}, $k \mathbb Z_n$ is an additive subgroup of $\mathbb Z_n$.
Hence, we only need to show that for each $a\in \mathbb Z_n$ and each $x\in k \mathbb Z_n$ their product $a x \pmod n$ is in $k \mathbb Z_n$.
Suppose $a x = r \pmod n$.
Then $a x = q n + r$.
Since $k|x$ and $k | n$, $r$ is divisible by $k$.
The latter means that $r$ belongs to $k \mathbb Z_n$ and we are done.

Let $I\subseteq \mathbb Z_n$ be any ideal.
By Claim~\ref{claim::Znsubgroups}, every subgroup of $\mathbb Z_n$ is of the form $k \mathbb Z_n$ for some $k | n$.
Since $I$ must be subgroup with respect to addition, $I$ must be of the form $k \mathbb Z_n$.
\end{proof}



\subsection{Homomorphisms of Rings}

We used homomorphisms of groups in order to compare different groups and isomorphisms provides us with a way of saying that to groups are the same.
We can extend this approach to the case of rings.

\begin{definition}
Let $(R, +, \cdot)$ and $(S, +, \cdot)$ be rings.
We are going to define a homomorphism $\phi\colon R\to S$.
\begin{itemize}
\item\textbf{Data:} 
\begin{enumerate}
\item A map $\phi\colon R\to S$.
\end{enumerate}

\item\textbf{Axioms:}
\begin{enumerate}
\item $\phi(a + b) = \phi(a) + \phi(b)$ for all $a, b\in R$.

\item $\phi(ab) = \phi(a)\phi(b)$ for all $a,b\in R$.

\item $\phi(1) = 1$.
\end{enumerate}
\end{itemize}
In this case, we say that $\phi$ is a homomorphism from $R$ to $S$.
If in addition we have
\begin{itemize}
\item[]
\begin{enumerate}
\setcounter{enumi}{3}
\item $\phi$ is bijective
\end{enumerate}
\end{itemize}
then $\phi$ is called an isomorphism.
In this case $R$ and $S$ are called isomorphic.
\end{definition}

As before, this is not the most general notion of a homomorphism.
But I am going to stick to this convenient and commonly used case.

\begin{remarks}
\begin{enumerate}
\item It should be noted that if $\phi\colon R\to S$ is a homomorphism of rings, then at least $\phi\colon (R,+) \to (S, +)$ is a homomorphism of abelian groups.
In particular, $\phi(0) = 0$ and $\phi(-a) = - \phi(a)$ by Claim~\ref{claim::HomGrProp}.

\item If $R$ and $S$ are isomorphic rings, then they are basically the same.
We already discussed how to understand isomorphisms in case of arbitrary groups, see discussion after Definition~\ref{def::IsomorphismGr}.
Briefly, isomorphism rename elements of $R$ into elements of $S$ and switch addition and multiplication of $R$ into addition and multiplication of $S$, respectively.
Isomorphic rings have exactly the same properties.
\end{enumerate}
\end{remarks}

\begin{examples}
\begin{enumerate}
\item The map $\mathbb Z\to \mathbb Z_n$ via $k\mapsto k \pmod n$ is a ring homomorphism.

\item The map $\mathbb R\to \operatorname{M}_n(\mathbb R)$ via $\lambda \mapsto \lambda E$, where $E$ is the identity matrix, is a ring homomorphism.

\item The map $\mathbb R[x]\to \mathbb C$ via $f(x) \mapsto f(i)$, where $i^2 = -1$, is a ring homomorphism.

\item The map $\mathbb C\to \operatorname{M}_n(\mathbb R)$ via $a + bi \mapsto \left(\begin{smallmatrix}{a}&{-b}\\{b}&{a}\end{smallmatrix}\right)$ is a ring homomorphism.
\end{enumerate}
\end{examples}


\begin{claim}
[The Chinese Remainder Theorem]
Let $n$ and $m$ be coprime natural numbers, that is $(n,m) = 1$.
Then the map
\[
\Phi \colon \mathbb Z_{mn} \to \mathbb Z_m \times \mathbb Z_n,\quad k\mapsto (k\!\!\mod m, k\!\!\mod n)
\]
is a ring isomorphism.
\end{claim}
\begin{proof}
We already know that $\Phi\colon (\mathbb Z_{mn}, +) \to (\mathbb Z_m\times\mathbb Z_n, +)$ is an isomorphism of abelian groups, see Claim~\ref{claim::Chinese}.
Also, we checked that $\Phi$ preserves the multiplication and the identity element, see the proof of Claim~\ref{claim::ChineseMult}.
\end{proof}


\begin{definition}
Let $\phi\colon R\to S$ be a homomorphism of rings.
Then
\begin{itemize}
\item The kernel of $\phi$ is $\ker\phi = \{r\in R\mid \phi(r) = 0\}\subseteq R$.

\item The image of $\phi$ is $\Im \phi = \{\phi(r) \mid r\in R\} = \phi(R)\subseteq S$.
\end{itemize}
\end{definition}

\begin{claim}
\label{claim::RingHomProp}
Let $\phi\colon R\to S$ be a homomorphism of rings.
Then
\begin{enumerate}
\item $\Im\phi\subseteq S$ is a subring.

\item $\ker \phi\subseteq R$ is an ideal.

\item The map $\phi$ is surjective if and only if $\Im\phi = S$.

\item The map $\phi$ is injective if and only if $\ker \phi = \{0\}$.
\end{enumerate}
\end{claim}
\begin{proof}
1) We already know that $\Im \phi$ is an additive subgroup, see Claim~\ref{claim::HomProp}.
By definition of the homomorphism $1 = \phi(1) \in \Im \phi$.
Hence, we need to show that it is closed under multiplication.
Indeed, if $x,y\in \Im\phi$, then $x = \phi(a)$ and $y = \phi(b)$ for some $a,b\in R$.
Then,
\[
xy = \phi(a) \phi(b) = \phi(ab) \in \Im \phi
\]

2) We already know that $\ker \phi$ is an additive subgroup, see Claim~\ref{claim::HomProp}.
Hence, we should show that it is stable under multiplication by any element of $R$.
Let $x\in \ker \phi$ and $r\in R$, we need to show that $rx, xr \in \ker \phi$, that is $\phi(rx) = 0$ and $\phi(xr) = 0$.
Indeed, $\phi(rx) = \phi(r)\phi(x) = \phi(r) 0 = 0$ and similarly $\phi(xr) = 0$.

3) The captain Obvious told us that this is true.

4) Since $\phi\colon (R,+)\to (S,+)$ is a homomorphism of additive groups, the result follows from Claim~\ref{claim::HomProp}.

\end{proof}

