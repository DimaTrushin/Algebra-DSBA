\ProvidesFile{lecture09.tex}[Lecture 9]


\subsection{The Buchberger Criterion}

There is another hidden problem.
Is there at least one Gr\"obner basis?
Or how to check if $G$ is a Gr\"obner basis?
We are going to address the problem with the Buchberger criterion.
I am going to postpone the proof until the next lecture and to focus on the criterion with its applications.

\begin{definition}
Suppose $F$ is a field, $f_1,f_2\in F[x_1,\ldots,x_n]$ are some nonzero polynomials, and we are given a lexicographical order on monomials.
Assume that $f_1 = c_1 m_1 + f'_1$, where $c_1m_1$ is the leading term, and $f_2 = c_2 m_2 + f'_2$, where $c_2m_2$ is the leading term.
Let $m$ be the least common multiple of $m_1$ and $m_2$, then $m = m_1 t_1 = m_2 t_2$.
Then, the polynomial
\[
S_{f_1\,f_2} = c_2t_1 f_1 - c_1 t_2 f_2 = c_2t_1 f'_1 - c_1 t_2 f'_2
\]
is called S-polynomial of $f_1$ and $f_2$.
\end{definition}


\begin{example}
\label{example::SPoly}
Consider $\mathbb Q[x, y, z]$, $f_1 = xy - 1$, $f_2 = yz - 1$, and $\operatorname{Lex}(x, y, z)$ is the order.
Then, the leading terms are $m_1 = xy$ and $m_2 = yz$ and the least common multiple is $xyz = m_1 z = m_2 x$.
 Then,
\[
S_{f_1\,f_2} = z f_1 - x f_2 = z(xy - 1) - x(yz - 1) = x-z
\]
\end{example}


\begin{claim}
[The Buchberger criterion]
Suppose $F$ is a field, we are given a lexicographical order on monomials in $n$ variables, and $G\subseteq F[x_1,\ldots,x_n]\setminus\{0\}$ is any set.
Then, the following conditions are equivalent
\begin{enumerate}
\item $G$ is a Gr\"obner basis.

\item For every $g_1,g_2\in G$, $S_{g_1\,g_2}$ is reducible to zero with respect to $G$.%
\footnote{There is at least one way to reduce each S-polynomial to zero.}
\end{enumerate}
\end{claim}

\begin{remark}
In Example~\ref{example::SPoly}, we see that the S-polynomial of $f_1$ and $f_2$ does not reduce to zero with respect to $\{f_1, f_2\}$.
Hence, the set is not a Gr\"obner basis.
\end{remark}

There is a special case ensuring that an S-polynomial will always reduce to zero.%
\footnote{There is one additional trick but it is quite complicated and not worth mentioning.}

\begin{claim}
Suppose $F$ is a field, $g_1,g_2\in F[x_1,\ldots,x_n]\setminus\{0\}$, and we are given a lexicographical order.
Assume that the leading monomials of $g_1$ and $g_2$ are coprime.
Then, $S_{g_1\,g_2}$ reduces to zero with respect to $\{g_1,g_2\}$.
\end{claim}
\begin{proof}
Suppose $g_1 = c_1 m_1 + g_1'$ and $g_2 = c_2 m_2 + g_2'$, where $m_i$ is the leading monomial of $g_i$, $c_i$ is the leading coefficient of $g_i$ and $g_i'$ is the tail of $g_i$.
Since $m_1$ and $m_2$ are coprime the $S$ polynomial of $g_1$ and $g_2$ is
\[
S_{g_1\,g_2} = c_2m_2 g_1 - c_1 m_1 g_2 = c_2m_2 (c_1 m_1 + g_1') - c_1 m_1 (c_2 m_2 + g_2') = c_2m_2 g_1' - c_1 m_1 g_2'
\]
Now we use equality $c_i m_i = g_i - g_i'$ and get
\[
S_{g_1\,g_2} = c_2m_2 g_1' - c_1 m_1 g_2' = (g_2 - g_2')g_1' - (g_1 - g_1')g_2' = g_2' g_1 - g_1'g_2
\]
So, we expressed the S-polynomial in the form $a g_1 + b g_2$ such that each monomial of $a$ is strictly smaller than $m_2$ and each monomial of $b$ is strictly smaller than $m_1$.

Now the goal is to show that the polynomial $a g_1 + b g_2$ is either zero or we can reduce it at least once with $\{g_1, g_2\}$.
In order to do that, we show that the leading monomial $M(a g_1 + b g_2)$ is either $M(a g_1)$ or $M(bg_2)$.
If $M(a g_1)> M(bg_2)$, then $M(a g_1)$ is the leading monomial of $a g_1 + b g_2$.
If $M(a g_1) < M(bg_2)$, then $M(bg_2)$ is the leading monomial of $a g_1 + b g_2$.
So, we may assume that $M(a g_1) = M(bg_2)$.
However, $m_1 = M(g_1)$ divides $M(a g_1)$ and $m_2 = M(g_2)$ divides $M(b g_2)$.
Since the $m_1$ and $m_2$ are coprime, $m_1 m_2$ divides $M(a g_1) = M(a) M(g_1) = M(a) m_1$.
Therefore $m_2$ divides $M(a)$.
This contradicts to the fact that all monomials of $a$ are smaller then $m_2$.
Hence, $a$ must be zero.
The same argument shows that $b$ is zero also.
Now assume that $a g_1 + b g_2$ is not zero.
Then its leading monomial is either $M(a) m_1$ or $M(b) m_2$.
In either case, it is reducible with respect to $g_1$ or $g_2$.

Let us go back to the S-polynomial of $g_1$ and $g_2$.
We rewrote the S-polynomial as follows $a g_1 + b g_2$ such that monomials of $a$ are smaller than $m_2$ and monomials of $b$ are smaller than $m_1$.
In the previous paragraph we showed that the leading monomial of $a g_1 + b g_2$ is either $M(a) m_1$ or $M(b) g_2$.
Without loss of generality, we may assume that the leading term is $M(a) m_1$.
Let us reduce the leading term using $g_1$.
\[
S_{g_1, g_2}\stackrel{f}{\longrightarrow} a g_1 + b g_2 - C(a) M(a) g_1 = (a - C(a) M(a)) g_1 + b g_2
\]
As we can see, the result of reduction is also a polynomial of the form $a g_1 + b g_2$ such that all monomials of $a$ are smaller than $m_2$ and all monomials of $b$ are smaller than $m_1$.
Hence, we can repeat the reduction process.
However, by Claim~\ref{claim::ReductFin} the process must terminate.
But the latter means that the result of the reduction is zero (otherwise we would be able to proceed with the reduction).
This completes the proof.
\end{proof}

\begin{example}
Suppose we are given $\mathbb Q[x, y, z]$ with $\operatorname{Lex}(x, y, z)$, $g_1 = x^2 y - z$, and $g_2 = z^2 + 1$.
Then the leading monomials $M(g_1) = x^2 y$ and $M(g_2) = z^2$ are coprime.
Hence, their S-polynomial reduces to zero with respect to $\{g_1,g_2\}$.
We can check this manually, this is not hard.
But it is much easier to apply the claim and avoid computation at all.
\end{example}


\subsection{Ideals in polynomial rings}

It turns out that a lot of problems about polynomials can be solved using ideals.
Hence, it is worth talking about ideals in a polynomial ring in several variables for a bit.

\begin{definition}
Suppose $F$ is a field and we are given a finite set of polynomials $g_1,\ldots,g_k\in F[x_1,\ldots,x_n]$.
Then the set
\[
(g_1,\ldots,g_k) = \{g_1 h_1 + \ldots + g_k h_k\mid h_1,\ldots,h_k\in F[x_1,\ldots,x_n]\}
\]
is an ideal of $F[x_1,\ldots,x_n]$ and is called the ideal generated by $g_1,\ldots,g_k$.
If take $G = \{g_1,\ldots,g_k\}$, then the ideal $(g_1,\ldots,g_k)$ is also denoted by $(G)$ for short.
\end{definition}

There is a nontrivial result saying that every ideal of $F[x_1,\ldots,x_n]$ is generated by a finite set.
The result was obtained by Hilbert and is known as the Hilbert basis theorem.

\begin{claim}
[The Hilbert Basis Theorem]
Suppose $F$ is a field and $I\subseteq F[x_1,\ldots,x_n]$ is an ideal.
Then there are polynomials $g_1,\ldots,g_k\in F[x_1,\ldots,x_n]$ such that $I = (g_1,\ldots,g_k)$.
\end{claim}

If we want to solve problems about ideals effectively, we need to generate the ideals by Gr\"obner bases.
Hence, we need to solve the following problem.
Suppose an ideal $I$ is generated by $g_1,\ldots,g_k$ and we fix a lexicographical order.
It may happen that the initial set $\{g_1,\ldots,g_k\}$ is not a Gr\"obner basis.
Hence, we want to replace it by a different set $G$ such that $G$ is a Gr\"obner basis and generates $I$, that is, $I = (G)$.
Additionally, we want $G$ to be finite, otherwise it is almost useless from the effectiveness point of view.
The latter problem can be solved by the Buchberger algorithm.


\paragraph{The Buchberger Algorithm}

As usual, we have a field $F$, a polynomial ring $F[x_1,\ldots,x_n]$, and we are given a lexicographical order on monomials.

\paragraph{Input}

A finite set of polynomials $G = \{g_1,\ldots,g_k\}\subseteq F[x_1,\ldots,x_n]$.

\paragraph{Output}

A finite set of polynomials $G_0\subseteq F[x_1,\ldots,x_n]$ such that
\begin{enumerate}
\item $G_0$ is a Gr\"obner basis.

\item $(G) = (G_0)$.
\end{enumerate}

\paragraph{Algorithm}

\begin{enumerate}
\item At the beginning we initialize $G_0 = G$.

\item For each $g_i,g_j\in G_0$ we compute $S_{g_i\,g_j}$ and reduce it by $G_0$ to a remainder $S_{g_i\,g_j}\stackrel{G_0}{\rightsquigarrow}r_{ij}$.

\item If all $r_{ij}$ are zero, then $G_0$ is the required Gr\"obner basis.
If not, then we update $G_0$ to be $G_0\cup \{r_{ij}\mid r_{ij} \neq 0\}$ and repeat the step~(2).
\end{enumerate}

If we want to show that the algorithm works, then we need to show several things: 1) it halts at a finite step, 2) the set $G_0$ is finite, 3) the resulting set $G_0$ is a Gr\"obner basis, 4) the resulting set generates the same ideal as $G$.
The most difficult item is the first one.
I am going to postpone the proof until the next lecture and explain all the other items.

2) Suppose the loop in the algorithm worked $n$ times.
Then, we constructed a set $G_1$ from the set $G$, then the set $G_2$, next $G_3$ and so on.
At the end we terminated at the set $G_n$.
As you can see at each step the set $G_k$ is obtained from the set $G_{k-1}$ by adding finitely many polynomials.
Hence, all sets $G_k$ are finite.
In particular, the final set $G_0 = G_n$ is finite.

3) Since algorithm terminates on $G_n$, all S-polynomials of elements of $G_n$ reduces to zero with respect to $G_n$.
Hence, the Buchberger Criterion shows that $G_n$ is a Gr\"obner basis.

4) We need to show that $(G) = (G_n)$.
It is enough to show that on each step $(G_{k-1}) = (G_k)$.
But $G_k = G_{k-1} \cup \{r_{ij}\mid r_{ij}\neq 0\}$.
In particular, $G_{k-1}\subseteq G_k$ and thus $(G_{k-1}) \subseteq (G_k)$.
In order to show the other inclusion it is enough to show that each $r_{ij}$ belongs to $(G_{k-1})$.
Indeed, for each $g_i,g_j\in G_{k-1}$ the S-polynomial $S_{g_i\,g_j}$ is of the form $c_2t_1 g_i - c_1 t_2 g_j\in (G_{k-1})$ by definition.
Since, $S_{g_i\,g_j}\stackrel{G_{k-1}}{\rightsquigarrow}r_{ij}$.
This means there exist a sequence of reductions: $S_{g_i\,g_j}\stackrel{g_1'}{\longrightarrow}f_1\stackrel{g_2'}{\longrightarrow}f_2\stackrel{g_3'}{\longrightarrow}\ldots\stackrel{g_k'}{\longrightarrow}f_k = r_{ij}$.
Hence, it is enough to show that if $f\in (G)$, $g\in G$, and $f\stackrel{g}{\longrightarrow}f'$, then $f'\in (G)$.
But this is clear by the definition, because $f' = f - \lambda t g$ for some $\lambda \in F$ and a monomial $t$.

\subsection{Rings of remainders}

Suppose $F$ is a field, we consider $F[x_1,\ldots,x_n]$, and a lexicographical order is fixed.
Suppose $G = \{g_1,\ldots,g_k\}\subseteq F[x_1,\ldots,x_n]$ and $I = (G)$.
We compute a Gr\"obner basis of $I$ and denote it by $G_0$.
Hence, we also have $I = (G_0)$.

Now I want to define the ring of remainders using the Gr\"obner basis $G_0$ as follows.
As a set
\[
F[x_1,\ldots,x_n]/(G_0) = \{f\in F[x_1,\ldots,x_n]\mid f \text{ is a remainder with respect to }G_0\}
\]
The addition is the usual addition of polynomials.
The multiplication is defined as follows: suppose $f, f'\in F[x_1,\ldots,x_n]/(G_0)$, we compute the usual product of $f$ and $f'$ as polynomials and reduce it with respect to $G_0$, that is $f f' \stackrel{G_0}{\rightsquigarrow} r$.
Then, we define $f \cdot f' = r$ in $F[x_1,\ldots,x_n]/(G_0)$.
One can show that we have defined a commutative ring.

\begin{remarks}
By the definition of the remainder ring $F[x_1,\ldots,x_n]/(G_0)$ its construction depends on the choice of the Gr\"obner basis $G_0$.
In particular, it depends on the choice of the lexicographical order.
It turns out that the remainder ring depends only on the choice of the ideal $I = (G_0)$ in the following sense: 
\begin{enumerate}
\item Let us fix a lexicographical order and find two different Gr\"obner bases of the ideal $I$, say, $G_1$ and $G_2$.
Suppose $f\in F[x_1,\ldots,x_n]$ and $f\stackrel{G_1}{\rightsquigarrow}r_1$ and $f\stackrel{G_2}{\rightsquigarrow}r_2$, where $r_1$ and $r_2$ are remainders with respect to $G_1$ and $G_2$.
Then, one can show that $r_1 = r_2$.
In particular, the sets of remainders with respect to $G_1$ and $G_2$ are the same
\[
F[x_1,\ldots,x_n]/(G_1) = F[x_1,\ldots,x_n]/(G_2)
\]
Moreover, this also implies that the operations are the same.
Hence, the rings for different Gr\"obner bases are exactly the same.

\item Now suppose that we take two different lexicographical orders $\Lex_1$ and $\Lex_2$ and compute Gr\"obner bases $G_1$ and $G_2$ of $I$ corresponding to $\Lex_1$ and $\Lex_2$, respectively.
Then, the sets of remainders can be different.
However, the rings of remainders will be isomorphic.
\[
F[x_1,\ldots,x_n]/(G_1) \simeq F[x_1,\ldots,x_n]/(G_2)
\]
The isomorphisms are given by the following rules.
If $f\in F[x_1,\ldots,x_n]/(G_1)$, we compute its remainder with respect to $G_2$, that is $f\stackrel{G_2}{\rightsquigarrow}f'$.
Then the map $f \mapsto f'$ is an isomorphism of rings.
The inverse map is given as follows.
We take $g\in F[x_1,\ldots,x_n]/(G_2)$ and compute its remainder with respect to $G_1$, that is  $g\stackrel{G_1}{\rightsquigarrow}g'$.
Then, the map $g\mapsto g'$ is the inverse isomorphism.
\end{enumerate}
\end{remarks}

\subsection{Membership problem and variable elimination}

There are two very popular problems that can be solved using Gr\"obner bases.
I want to discuss them both.

\paragraph{Membership problem}

Suppose we have a field $F$, a polynomial ring $F[x_1,\ldots,x_n]$, an ideal $I = (g_1,\ldots,g_k)$ in the polynomial ring, and a polynomial $f$.
The question is how to check if $f$ belongs to $I$.
Here is an algorithm.
\begin{enumerate}
\item Fix a lexicographical order $\Lex$.

\item Find a Gr\"obner basis $G$ of $I$.

\item Compute the remainder of $f$ with respect to $G$, that is $f\stackrel{G}{\rightsquigarrow}r$.

\item The polynomial $f$ belongs to $I$ if and only if $r = 0$.
\end{enumerate}

It should be noted that if $G$ is not a Gr\"obner basis, then only a half of the algorithm works correctly.
If the remainder $r$ is zero, then $f$ is guaranteed to be in $I$.
The proof is similar to the one we used to show that the Buchberger algorithm is correct.
However, in case the remainder $r$ is not zero the polynomial $f$ may still be in $I$.
Indeed, in Example~\ref{example::SPoly} S-polynomial $S_{f_1\,f_2}$ belongs to the ideal $(f_1, f_2)$.
However, $S_{f_1\,f_2}$ is not reducible with respect to $\{f_1,f_2\}$.

\paragraph{Variable elimination}

Suppose we have a field $F$, a polynomial ring $F[x_1,\ldots,x_n, x_{k+1}, \ldots, x_n]$, and an ideal $I = (g_1,\ldots,g_k)$ in the polynomial ring.
The question is how to compute $I\cap F[x_1,\ldots,x_k]$.
This means, we want to eliminate all polynomials from $I$ depending on $x_{k+1},\ldots,x_n$.
Here is an algorithm.
\begin{enumerate}
\item  Fix a lexicographical order such that $x_1,\ldots,x_k < x_{k+1},\ldots,x_n$.
The ordering of $x_1,\ldots,x_k$ or $x_{k+1},\ldots,x_n$ does not matter.

\item Compute a Gr\"obner basis $G_0$ of $I$.

\item Define $G = \{g\in G_0\mid g \text{ does not depend on } x_{k+1},\ldots,x_n\}$.

\item Then $G$ is a Gr\"obner basis of $I\cap F[x_1,\ldots,x_k]$ with respect to the lexicographical order given by the chosen ordering of $x_1,\ldots,x_k$.
\end{enumerate}
