\ProvidesFile{lecture03.tex}[Lecture 3]


\subsection{Homomorphisms and Isomorphisms}

We have a lot of different groups in algebra.
And we will study a couple of ways to produce new groups from given ones.
In this situation, we need a way to compare the groups.
How to recognize that we have constructed the same group that we already know?
In order to answer this question, we need to explain what it means that two groups are the same.
So, we need a way to compare two groups.
This leads us to the notions of homomorphism (a way to compare groups) and isomorphism (a way to say that groups are the same).
Let me proceed with formal definitions.

\begin{definition}
Let $G$ and $H$ be groups.
We define a homomorphism $\varphi\colon G\to H$.
\begin{itemize}
\item \textbf{Data} A map $\varphi\colon G\to H$.

\item \textbf{Axiom} $\varphi(g_1g_2) = \varphi(g_1) \varphi(g_2)$ for any $g_1,g_2\in G$.
\end{itemize}
In this case $\varphi$ is called a homomorphism from $G$ to $H$.
\end{definition}

\begin{remark}
Let me explicitly repeat the definition.
We are given a group $(G, \circ)$ and $(H,\cdot)$.
A homomorphism $\varphi \colon G\to H$ is a map such that $\varphi(g_1 \circ g_2) = \varphi(g_1) \cdot \varphi(g_2)$.
On the left-hand side we take elements $g_1$ and $g_2$ from $G$ and multiply them using the operation on $G$ and then send the resulting element to $H$.
On the right-hand side, we send the elements $g_1$ and $g_2$ to the group $H$ first and then multiply the images using operation on $H$.
\end{remark}

\begin{examples}
\begin{enumerate}
\item Let $G = (\mathbb Z, +)$ and $H = (\mathbb Z_n, +)$, then the map $\pi\colon \mathbb Z\to \mathbb Z_n$ by the rule $k \mapsto k \pmod n$ is a homomorphism.

\item Let $G = S_n$ be the group of permutations and $H = \mu_2 = \{\pm 1\}$ with multiplication.
Then the map $\sgn \colon S_n \to \mu_2$ taking each permutation to its sign (even one goes to $1$ and odd one goes to $-1$) is a homomorphism.

\item Let $G = (\operatorname{GL}_n(\mathbb R),\cdot)$ and $H = (\mathbb R^*, \cdot)$ be the set of non-zero real numbers with multiplication.
Then the map $\det \colon \operatorname{GL}_n(\mathbb R) \to \mathbb R^*$ by the rule $A \mapsto \det(A)$ is a homomorphism.

\item Let $G = (\mathbb R, +)$ and $H = (\mathbb R^*, \cdot)$.
Then the map $\exp\colon \mathbb R\to \mathbb R^*$ by the rule $x \mapsto e^x$ is a homomorphism.

\item Let $G = (\mathbb Z, +)$, $H$ be an arbitrary group and $h\in H$ be an arbitrary element.
Then the map $\phi\colon \mathbb Z \to H$ by the rule $k \mapsto h^k$ is a homomorphism.

\item Let $G = (\mathbb Z_n, +)$, $H$ be an arbitrary group and $h\in H$ be an element such that $h^n = 1$.
Then the map $\phi\colon \mathbb Z_n \to H$ by the rule $k \mapsto h^k$ is a group homomorphism.
\end{enumerate}
\end{examples}

Let us prove several properties of homomorphisms.

\begin{claim}
\label{claim::HomGrProp}
Let $\varphi\colon G\to H$ be a homomorphism of groups.
Then
\begin{enumerate}
\item $\varphi(1) = 1$, that is the neutral element of $G$ goes to the neutral element of $H$.

\item $\varphi(g^{-1}) = \varphi(g)^{-1}$ whenever $g\in G$.
\end{enumerate}
\end{claim}
\begin{proof}
1) We know that $1 = 1 \cdot 1$.
Let us apply $\varphi$ to this equality.
Then we get
\[
\varphi(1) = \varphi(1 \cdot 1) = \varphi(1) \varphi(1) \in H
\]
Now, multiply this equality by $\varphi(1)^{-1}$, we will get $1 = \varphi(1)$.


2) Let $g\in G$ be an arbitrary element.
Then, $g g^{-1} = 1$.
Let us apply $\varphi$ to this equality.
Then we get
\[
\varphi(g) \varphi(g^{-1}) = \varphi(g g^{-1}) = \varphi(1) = 1
\]
Now multiply this on the left by $\varphi(g)^{-1}$.
We will get $\varphi(g^{-1})  = \varphi(g)^{-1}$.
\end{proof}

\begin{definition}
\label{def::IsomorphismGr}
Let $G$ and $H$ be groups.
We define an isomorphism $\varphi \colon G\to H$.
\begin{itemize}
\item \textbf{Data} A homomorphism $\varphi \colon G\to H$.

\item \textbf{Axiom} $\varphi$ is bijective.
\end{itemize}
In this case, $\varphi$ is called an isomorphism between $G$ and $H$.
If there is an isomorphism between $G$ and $H$, the groups $G$ and $H$ are called isomorphic.
\end{definition}

Let me clarify the definition.
First let me explain what it means that $\varphi\colon X\to Y$ is a bijection (between sets).
Suppose $X = \{1, 2, 3\}$, $Y = \{a, b, c\}$, and $\varphi$ works as follows $1 \mapsto a$, $2\mapsto b$, and $3\mapsto c$.
Then you may think of it like this.
The set $X$ is a set of names for your elements and the set $Y$ is a set of some other names for your elements.
Than the map $\varphi$ is a renaming map it just switches the names of the elements.
Thus, you may think that $Y$ is the same set as $X$ but with elements named differently.

Now, if $\varphi\colon G\to H$ is an isomorphism of groups, then it is at least a bijection.
Hence, it identifies elements of $G$ and $H$ saying that the underlying sets of the groups are the same.
Also, the condition $\varphi(g_1g_2) = \varphi(g_1) \varphi(g_2)$ means that after this identification the operation on $G$ becomes an operation on $H$.
The latter means that you rename the elements and the operation.
Hence, you may think that $H$ is exactly the same group as $G$ but with different set of names for the elements and a different notation for the operation.
However, this is essentially the same group.
As a corollary, isomorphic groups have exactly the same properties.

\begin{examples}
\begin{enumerate}
\item Let $G = (\mathbb Z_n, +)$ and $H = \mu_n\subseteq \mathbb C$ be the set of complex roots of unity with multiplication as an operation.
Let us fix a primitive root $\xi \in\mu_n$.
Then the map $\mathbb Z_n \to \mu_n$ by the rule $k \mapsto \xi^k$ is an isomorphism.

\item Let $G = (\mathbb Z, +)$ and 
\[
H = \left\{\left.
\begin{pmatrix}
{1}&{n}\\
{0}&{1}
\end{pmatrix}
\;\right|\;
n\in \mathbb Z
\right\}
\]
with multiplication as an operation.
Then the map $\varphi\colon \mathbb Z\to H$ by the rule $k \mapsto \left(\begin{smallmatrix}{1}&{k}\\{0}&{1}\end{smallmatrix}\right)$ is an isomorphism.

\item Let $G = (\mathbb C, +)$ and $H = (\mathbb R^2, +)$.
Then the map $\varphi \colon \mathbb C\to \mathbb R^2$ by the rule $z \mapsto (\Re z, \Im z)$ is an isomorphism.

\item Let $G = (\mathbb C^*, \cdot)$ and 
\[
H = \left\{\left.
\begin{pmatrix}
{a}&{-b}\\
{b}&{a}
\end{pmatrix}
\;\right|\;
a, b\in \mathbb R\text{ such that }a^2 + b^2 \neq 0
\right\}
\]
with multiplication as an operation.
Then the map $\varphi \colon \mathbb C^*\to H$ by the rule  $a + bi \mapsto \left(\begin{smallmatrix}{a}&{-b}\\{b}&{a}\end{smallmatrix}\right)$ is an isomorphism.

\item Claim~\ref{claim::CyclicClass} says that a cyclic group $G = \langle g \rangle$ is isomorphic to $\mathbb Z$ or $\mathbb Z_n$ depending on the order of a generator.
If $\ord g = \infty$, then $G\simeq \mathbb Z$.
If $\ord g = n$, then $G\simeq \mathbb Z_n$.
\end{enumerate}
\end{examples}

With each homomorphism we may associate several subgroups: kernel and image.

\begin{definition}
Let $\varphi\colon G\to H$ be a homomorphism of groups.
Then
\begin{enumerate}
\item The kernel of $\varphi$ is $\ker \varphi = \{g\in G \mid \varphi(g) = 1\} \subseteq G$.

\item The image of $\varphi$ is $\Im \varphi = \{\varphi(g)\mid g\in G\} = \varphi(G) \subseteq H$.
\end{enumerate}
\end{definition}

It should be noted that the kernel is a subset of $G$ (belongs to the source of the map $\varphi$) and the image is a subset of $H$ (belongs to the target of the map $\varphi$).

\begin{claim}
\label{claim::HomProp}
Let $\varphi\colon G\to H$ be a homomorphism of groups.
Then
\begin{enumerate}
\item $\Im \varphi \subseteq H$  is a subgroup.

\item $\ker \varphi \subseteq G$ is a normal subgroup.

\item The map $\varphi$ is surjective if and  only if $\Im \varphi = H$.

\item The map $\varphi$ is injective if and only if $\ker \varphi  = \{1\}$.
\end{enumerate}
\end{claim}
\begin{proof}
1) Let us check all the requirements for being subgroup.
First, $1 = \varphi(1) \in \Im \varphi$, hence we have the identity.
Second, $\varphi(g_1)\varphi(g_2) = \varphi(g_1 g_2) \in \Im\varphi$, thus it is closed under the operation.
Third, $\varphi(g)^{-1} = \varphi(g^{-1}) \in \Im\varphi$, therefore it contains the inverse element for every element.

2) Let us check the requirements for the subgroup.
First, $\varphi(1) = 1$, hence $1 \in \ker \varphi$ by definition.
Second, if $x, y\in \ker \varphi$, then $\varphi(xy) = \varphi(x) \varphi(y) = 1\cdot 1 = 1$.
Therefore, $xy\in \ker\varphi$.
Third, if $x\in \ker\varphi$, then $\varphi(x^{-1}) = \varphi(x)^{-1} = 1^{-1} = 1$.
Hence, $x^{-1}\in \ker\varphi$.
We have just verified that $\ker \varphi$ is a subgroup.
We should show that $g \ker \varphi = \ker \varphi g$ for every $g\in G$.
By Claim~\ref{claim::normal_crit}, we need to show that $g\ker \varphi g^{-1}\subseteq \ker \varphi$ for each $g\in G$.
That is we should show that $\varphi(g \ker \varphi g^{-1}) = 1$ for each $g\in G$.
Indeed, for each $h\in \ker \varphi$, we have
\[
\varphi(g h g^{-1}) = \varphi(g) \varphi(h) \varphi(g^{-1}) = \varphi(g) \cdot 1 \cdot \varphi(g^{-1}) = \varphi(g) \varphi(g^{-1}) = \varphi(g g^{-1}) = \varphi(1) = 1
\]


3) This holds trivially by the definition.

4) Suppose $\varphi$ is injective and $x\in \ker\varphi$.
The latter means that $\varphi(x) = 1$.
From the other hand, we know that $\varphi(1) = 1$.
Hence $x$ and $1$ go to the same element $1$.
This means that $x = 1$ by the injectivity.

Now suppose that $\ker \varphi = \{1\}$.
Consider two elements $x, y\in G$ such that $\varphi(x) = \varphi(y)$.
Multiplying by $\varphi(x)^{-1}$, we get
\[
1 = \varphi(y) \varphi(x)^{-1} = \varphi(y) \varphi(x^{-1}) = \varphi(yx^{-1})
\]
Hence $yx^{-1}\in\ker\varphi = \{1\}$.
Thus $y x^{-1} = 1$.
Therefore $y = x$ and we are done.
\end{proof}

\subsection{Product of groups}

In general, we do not want to produce new groups from scratch.
We want to construct a group using already given ones.
There are many different operations in algebra to produce new groups.
We are not going to learn all of them.
Instead, we discuss the simplest one, which is the most useful one at the same time.

\begin{definition}
Let $G$ and $H$ be groups, we define a new group $G\times H$ as follows
\begin{enumerate}
\item As a set it is a product of underlying sets of the groups: $G\times H = \{(g, h)\mid g\in G,\;h\in H\}$.

\item The operation 
\[
\cdot \colon (G\times H)\times (G\times H) \to G\times H
\]
is given by the rule
\[
(g_1, h_1) (g_2, h_2) = (g_1 g_2, h_1 h_2),\quad g_1, g_2,\in G, \;h_1, h_2\in H
\]
\end{enumerate}
The group $G\times H$ is called the product of the groups $G$ and $H$.
\end{definition}

It should be noted that we must show that $G\times H$ is indeed a group.
We have just defined the required data for a group.
However, we need to check the axioms.
Let me recall them.
\begin{itemize}
\item The operation is associative, that is
\[
(g_1, h_1)\Bigl( (g_2, h_2) (g_3, h_3) \Bigl) = \Bigl((g_1, h_1) (g_2, h_2)\Bigl) (g_3, h_3)
\]

\item There is an identity, $1 = (1, 1)$.

\item Each element has inverse,  $(g, h)^{-1} = (g^{-1}, h^{-1})$.
\end{itemize}
All the properties are verified by a direct computation.
I am leaving this as an exercise.
If we have several groups $G_1,\ldots,G_k$, we may produce the group $G_1\times \ldots \times G_k$ in a similar way.

\subsection{Finite Abelian Groups}

Now I want to focus on the most important class of groups, that is the class of finitely generated abelian groups.
Let me start with the definition.

\begin{definition}
A finite abelian group is a commutative (abelian) group $G$ with finitely many elements.
\end{definition}

The definition is not a surprise, the name of the term is clear enough.
But pay attention to the next result.

\begin{claim}
Let $G$ be a finite abelian group, the $G$ is isomorphic to a group $\mathbb Z_{n_1}\times \ldots \times \mathbb Z_{n_k}$.
\end{claim}

I am not going to prove this result.
The proof is not hard but requires some technical tools that we are not going to learn because of the lack of time.
Also the proof itself does not reveal any important technique.
So it is better to spend our time mastering the way of using the result instead of proving it.
Now, I want to show you several examples.

\begin{examples}
\begin{enumerate}
\item Let $G = \mathbb Z_8^*$ with multiplication as an operation.
It is obviously a finite abelian group, hence it must be a product of cyclic groups.
Indeed, we can check that
\[
\mathbb Z_8^* \simeq \mathbb Z_2\times \mathbb Z_2
\]
and the operation preserving bijection (that is isomorphism) is given by
\[
1 \leftrightarrow (0,0),\; 3 \leftrightarrow (1,0),\; 5\leftrightarrow(0,1),\;7\leftrightarrow(1,1)
\]
This is not the only way to identify these two groups.
We may take a different isomorphism
\[
1 \leftrightarrow (0,0),\; 3 \leftrightarrow (1,0),\; 7\leftrightarrow(0,1),\;5\leftrightarrow(1,1)
\]
I am not going to describe all possible ways, but it must be clear that there are many different isomorphisms serving our purpose.
Note also that the group is not cyclic because there is no element of order $4$ in it.

\item Let $G = \mathbb Z_9^*$ with multiplication as an operation.
This is also a finitely generated abelian group.
In this case, we have
\[
\mathbb Z_9^* \simeq \mathbb Z_6
\]
and there are two different isomorphisms
\[
\begin{aligned}
\mathbb Z_6 &\to \mathbb Z_9^*\\
k &\mapsto 2^k
\end{aligned}
\quad\text{and}\quad
\begin{aligned}
\mathbb Z_6 &\to \mathbb Z_9^*\\
k &\mapsto 5^k
\end{aligned}
\]
Also note that the group here is cyclic.
The elements $2$ and $5$ are different generators of the group.
The isomorphisms above correspond to the choice of a generator.
\end{enumerate}
\end{examples}


\begin{claim}
[The Chinese Remainder Theorem]
\label{claim::Chinese}
Let $m, n\in \mathbb N$ be two coprime positive integers, that is $(m, n) = 1$.
Then the map
\[
\Phi\colon \mathbb Z_{mn} \to \mathbb Z_m \times \mathbb Z_n,\quad
k \mapsto (k\!\!\mod m,\;k\!\!\mod n)
\]
is an isomorphism of groups.
\end{claim}
\begin{proof}
First, we should check that the map is a homomorphism.
We need to show that $\Phi(k + d) = \Phi(k) + \Phi(d)$, that is
\begin{gather*}
\Phi(k + d) = ( (k+d)\!\!\mod m,\;(k+d)\!\!\mod n) = ((k\!\!\mod m) + (d\!\!\mod m),\;(k\!\!\mod n) + (d\!\!\mod n)) =\\
= (k\!\!\mod m ,\;k\!\!\mod n) + (d\!\!\mod m ,\;d\!\!\mod n) = \Phi(k) + \Phi(d)
\end{gather*}

Now, I claim that the homomorphism is injective.
Claim~\ref{claim::HomProp} item~(4) ensures that it is enough to show that the kernel of the homomorphism consists of the identity element only.
By definition,
\[
\ker \Phi = \{k\in \mathbb Z_{mn}\mid k = 0\pmod m,\; k = 0\mod n\}
\]
Hence $k\in \ker \Phi$ if and only if $m$ divides $k$ and $n$ divides $k$.
Since $m$ and $n$ are coprime, this implies that $mn$ divides $ k$.
The latter means that $k = 0$ in $\mathbb Z_{mn}$.

In order to prove that $\Phi$ is an isomorphism, we need to show that it is surjective.
Let us compute the number of elements in both groups.
By definition $|\mathbb Z_{mn}| = mn$.
From the other hand, $|\mathbb Z_m\times \mathbb Z_n| = |\mathbb Z_m| \cdot |\mathbb Z_n| = m n$.
Hence, $\Phi$ is an injective map between two sets of the same size.
Hence, it must be bijective and we are done.
\end{proof}

From the previous claim it is clear how to take elements from $\mathbb Z_{mn}$ to the product $\mathbb Z_m\times \mathbb Z_n$.
However, It is worth mentioning who to produce the map in the other direction.
Since $m$ and $n$ are coprime, we have $1 = um + vn$ for some $u, v\in \mathbb Z$ by the Euclidean algorithm.
Now consider the element $a_1 = um = 1 - vn$.
It is clear that $a_1 \mapsto (0, 1)$ under the action of $\Phi$.
Similarly, the element $a_2 = vn = 1 - um$ goes to $(1, 0)$.
Hence, the element $(a, b)$ corresponds to the element $a a_1 + b a_2 \pmod{mn}$ in $\mathbb Z_{mn}$.

\begin{examples}
\begin{enumerate}
\item In case $m = 3$ and $n = 2$, we have $\mathbb Z_6\simeq \mathbb Z_3\times\mathbb Z_2$.
Here element $1$ goes to $(1, 1)$.
Hence $(1, 1)$ is the generator of the cyclic group $\mathbb Z_3\times \mathbb Z_2$.
Since $1 = 3 - 2$, we see that $3$ goes to $(0, 1)$ and $-2$ goes to $(1,0)$  (note that $- 2 = 4$ in $\mathbb Z_6$).
Hence the inverse map is given by $(a, b)\mapsto -2a + 3b = 4a + 3b\pmod 6$.

\item From the other hands, the group $\mathbb Z_2\times \mathbb Z_2$ is not cyclic.
Hence there is no isomorphism with $\mathbb Z_4$.

\item Another example of different presentations of a finite abelian group
\[
\mathbb Z_{30} \simeq \mathbb Z_6 \times \mathbb Z_5 \simeq \mathbb Z_3 \times\mathbb Z_{10}\simeq\mathbb Z_2 \times \mathbb Z_{15} \simeq\mathbb Z_2\times\mathbb Z_3\times \mathbb Z_5
\]
So, all five different constructions give us the same cyclic group.

\item In general, if $m = p_1^{k_1}\ldots p_r^{k_r}$, where $p_i$ are prime, then
\[
\mathbb Z_{m} = \mathbb Z_{p_1^{k_1}}\times \ldots \times \mathbb Z_{p_r^{k_r}}
\]
\end{enumerate}
\end{examples}

As we saw above, the same finite abelian group may be written in many different ways.
How to check quickly that two different representations give us the same group?
The answer is given in the next result.

\begin{claim}
\label{claim::FAGClass}
Let $G$ be a finite abelian group.
Then
\begin{enumerate}
\item $G$ is uniquely presentable in the following form
\[
G = \mathbb Z_{d_1}\times \ldots \times \mathbb Z_{d_k},\quad\text{where}\; 1 < d_1|d_2|\ldots|d_k\text{ are positive integers}
\]

\item $G$ is uniquely (up to permutation of factors) presentable in the following form
\[
G = \mathbb Z_{p_1^{k_1}}\times \ldots \times \mathbb Z_{p_r^{k_r}},\quad \text{where}\; p_i \text{ are (not necessarily distinct) primes},\; k_i\text{ are positive integers}
\]
\end{enumerate}
\end{claim}

It is important to mention that primes $p_i$ may repeat in the second presentation, that is $\mathbb Z_2\times \mathbb Z_4$ is one of the possible cases.

\begin{examples}
\begin{enumerate}
\item Let $G = \mathbb Z_2 \times \mathbb Z_6$ and $H = \mathbb Z_{12}$.
These groups are presented in the first form.
Since such a presentation is unique $G$ and $H$ are not isomorphic.

\item Let $G = \mathbb Z_2 \times \mathbb Z_6$ and $H = \mathbb Z_2\times \mathbb Z_2\times \mathbb Z_3$.
We see that $G$ is presented in the first form and $H$ is presented in the second one.
Let us recompute $G$ into the second form using the Chinese Remainder Theorem
\[
G = \mathbb Z_2 \times \mathbb Z_6 = \mathbb Z_2 \times (\mathbb Z_2 \times \mathbb Z_3) = H
\]
Hence, the groups are isomorphic.
\end{enumerate}
\end{examples}

