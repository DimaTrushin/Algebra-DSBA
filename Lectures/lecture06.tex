\ProvidesFile{lecture06.tex}[Lecture 6]


\section{Polynomials in one variable}

\subsection{Definition}

Let $F$ be a field.
A polynomial $f$ in a variable $x$ is a picture
\[
f = a_0 + a_1 x + \ldots + a_n x^n,\quad \text{where } a_i \in F
\]
Here operations $+$ and $\cdot$ are just symbols.
So, we take some elements $a_i$ of the field $F$ and write down a string of symbols as above on the sheet of paper.
So, formally a polynomial is just an ordered sequence of coefficients $(a_0,\ldots,a_n)$.
We also assume that coefficients $a_{n+1},a_{n+2},\ldots$ are zero for the polynomial $f$.
Hence, we may assume that there are countably many coefficients but only finitely many of them are nonzero.
This is very convenient if we want to compare polynomials or perform arithmetic operations.

Suppose we are given two polynomials
\[
f = a_0 + a_1 x + \ldots + a_n x^n\text{ and } g = b_0 + b_1 x + \ldots + b_m x^m
\]
We say that $f$ and $g$ are equal if $a_k = b_k$ for all $k\in \mathbb N$.%
\footnote{Here, I assume that $a_k = 0$ for $ k > n$ and similarly $b_k = 0$ for $k > m$.}
We can add and multiply polynomials using the following rules
\begin{gather*}
f = \sum_{k=0}^n a_k x^k\quad\text{and}\quad g = \sum_{k=0}^m x^k\\
f + g = \sum_{k \geqslant 0} (a_k + b_k) x^k\quad\quad
f g = \sum_{k\geqslant 0}\Bigl( \sum_{u + v = k} a_u b_v\Bigl) x^k
\end{gather*}
The set of all polynomials with coefficients in $F$ is denoted by $F[x]$.
A straightforward computation shows that $F[x]$ with the addition and multiplication defined above is a commutative ring.

\begin{remark}
It should be noted that each polynomial $f\in F[x]$ defines a function $\hat f\colon F\to F$ by the rule $a \mapsto f(a)$.
Namely, if $f = a_0 + a_1 x + \ldots + a_n x^n$, then $f(a) = a_0 + a_1 a + \ldots + a_n a^n$.
However, in general case $f$ is not uniquely determined by the function $\hat f$.
Indeed, suppose $F = \mathbb Z_2$.
Then, polynomials $f_n = x^n$ define exactly the same function $\hat f_n \colon \mathbb Z_2 \to \mathbb Z_2$ sending $0\mapsto 0$ and $1\mapsto 1$.
So, the polynomials $f_n$ are all distinct but they define the same function.
This explains why we give such a tricky definition of polynomials.
We just want them to be pictures and not functions.
And if we want, we can always produce a function for each polynomial.
\end{remark}

If $f\in F[x]$ is presented in the form $f = a_0 + a_1 x + \ldots + a_n x^n$, where $a_n \neq 0$, that is $n$ is the largest index of the nonzero coefficient present in $f$.
Then $n$ is called the degree of $f$ and is denoted by $\deg f$.
The coefficient $a_n$ is called the leading coefficient.
The degree of the zero polynomial is not well-defined.
We will assume that $\deg (0) = -\infty$.
A polynomial $f\in F[x]$ is an element of the field $F$ if and only if $\deg (f) \leqslant 0$ and $f$ is a nonzero constant if and only if $\deg f = 0$.%
\footnote{A constant here means an element of $F$.}

\begin{claim}
\label{claim::Degree}
Let $F$ be a field.
Then for any $f, g \in F[x]$, we have $\deg(fg) = \deg (f) + \deg(g)$.%
\footnote{It should be noted that we assume that $-\infty + a = a + (-\infty) = -\infty$.}
\end{claim}
\begin{proof}
If at least one of the polynomials is zero the required equality contains $-\infty$ on both sides and hence is true.
Therefore, we may assume that $f$ and $g$ are not zero.
Let 
\[
f = a_n x^n + a_{n-1}x^{n-1} + \ldots + a_1 x + a_0\quad \text{and} \quad g = b_mx^m + b_{m-1}x^{m-1} + \ldots + b_1 x + b_0
\]
such that $a_n\neq 0$ and $b_m\neq0$.
Then 
\[
fg = a_n b_m x^{m+n} + h(x)
\]
where $\deg h < m + n$.
Since $F$ is a field, $a_n b_m \neq 0$.
Hence, $\deg fg = n + m = \deg f + \deg g$.
\end{proof}

\begin{claim}
\label{claim::PolyZeroDiv}
Suppose $F$ is a field.
Then the only zero divisor of $F[x]$ is the zero polynomial.
\end{claim}
\begin{proof}
If $f, g\in F[x]$, $f\neq 0$, and $g\neq 0$, then $\deg(f) \geqslant 0$ and $\deg(g) \geqslant 0$.
Hence $\deg(fg) = \deg(f) + \deg(g)\geqslant 0$.
In particular, $fg \neq 0$.
\end{proof}

\begin{claim}
\label{claim::PolyInvert}
Suppose $F$ is a field.
Then a polynomial $f\in F[x]$ is invertible if and only if $f\in F^*$.
\end{claim}
\begin{proof}
If $f$ is invertible, then $fg = 1$ for some polynomial $g\in F[x]$.
Then, $0 = \deg(1) = \deg(fg) = \deg(f) + \deg(g)$.
Since the degree of non-zero polynomials is not negative, $\deg(f) = \deg(g) = 0$.
The latter means that $f$ and $g$ are constants.
\end{proof}


\subsection{Euclidean algorithm}

If $F$ is a field, then $F[x]$ has a division with remainder.
If $f, g \in F[x]$ and $g\neq 0$, then we may divide $f$ by $g$ with remainder.
The latter means that there exist unique $q, r\in F[x]$ such that $f = q g + r$ and $\deg(r) < \deg (g)$.
The polynomial $q$ is called the quotient and $r$ is the remainder.

Suppose $f, g\in F[x]$, we say that $f$ divides $g$ if $g = fh$ for some $h\in F[x]$.
It should be noted that any polynomial divides $0$.
Also, if $a\in F^*$, then $f$ divides $af$ and vise versa because $F$ is a field.

\begin{definition}
Suppose $F$ is a field.
A polynomial $f\in F[x]$ is called monic if its leading coefficient is equal to $1$.
\end{definition}

\begin{definition}
Let $F$ be a field and $f, g\in F[x]$ be some polynomials.
A polynomial $d\in F[x]$ is called a greatest common divisor of $f$ and $g$ if
\begin{enumerate}
\item $d$ divides both $f$ and $g$.

\item if $h$ divides both $f$ and $g$, then $h$ divides $d$.

\item $d$ is monic.
\end{enumerate}
\end{definition}


\begin{claim}
\label{claim::PolyIdeals}
Let $F$ be a field and $I\subseteq F[x]$ be an ideal.
Then $I = f F[x] = \{fh\mid h\in F[x]\}$ for some $f\in F[x]$.
\end{claim}
\begin{proof}
If $I$ consists of zero only, then $I = 0 F[x]$.
Let $f\in I$ be a non-zero polynomial of the least degree.
Then for every $h\in I$, we divide $h$ by $f$ with remainder and get $h = qf + r$, where $\deg r < \deg f$.
Then, $r = h - qf \in I$ and has the degree smaller than $f$.
Since $f$ is not zero and has the smallest possible degree, $r$ must be zero.
This completes the proof.
\end{proof}

The ideal $fF[x]$ is usually denoted by $(f)$ for short.
It is very convenient and I will stick to this notation.

\begin{claim}
\label{claim::PolyGCD}
Let $F$ be a field and $f,g\in F[x]$.
Then
\begin{enumerate}
\item There exist a greatest common divisor $d$ of $f$ and $g$ and polynomials $u,v\in F[x]$ such that $d = uf + v g$.

\item The greatest common divisor for $f$ and $g$ is unique.

\end{enumerate}
\end{claim}
\begin{proof}
(1) Consider the following set of polynomials
\[
I = \{a f + b g \mid a, b \in F[x]\}\subseteq F[x]
\]
This subset is an ideal of $F[x]$.
By Claim~\ref{claim::PolyIdeals}, there is a monic polynomial $r\in I$ such that $I = (r)$.
Since $r\in I$, $r = uf + vg$ for some $u, v\in F[x]$.
Since $f, g \in I = (r)$, $f = s r$ and $g = t r$ for some $s, t \in F[x]$.

Now, lets prove that $r$ is a greatest common divisor.
Equalities $f = s r$ and $g = t r$ mean that $r$ divides $f$ and $g$, that is $d$ is a common divisor.
If $h$ divides $f$ and $g$, then $h$ divides both summands in the right-hand side of  $r = uf + vg$.
Hence $h$ divides $r$.
All the properties of the gcd are satisfied.

(2) Suppose we have two gcds $d_1$ and $d_2$.
Then $d_1$ divides $d_2$ because $d_2$ is a gcd.
And vice versa, $d_2$ divides $d_1$ because $d_1$ is a gcd.
Thus
\[
d_1 = a d_2\quad d_2 = b d_1
\]
In particular, $d_1 = ab d_1$.
Hence, $d_1(1 - ab)= 0$ in $F[x]$.
But there is no zero-divisors in $F[x]$ by Claim~\ref{claim::PolyZeroDiv}.
Hence, $1 - ab = 0$ and, thus, $1 = ab$.
Then $a$ and $b$ are invertible elements of $F$.
\end{proof}

\begin{remarks}
\begin{itemize}
\item Let me explicitly state that if $f = g = 0$, then the greatest common divisor is $0$ by definition.
Indeed, in this case every polynomial divides $f$ and $g$ and $0$ is the divisor that is divisible by any other divisor.

\item If $f \neq 0$ and $g = 0$, then the greatest common divisor is $f$ divided by the leading coefficient because $f$ divides $g$ in this case.
\end{itemize}
\end{remarks}

% Euclidean algorithm, gcd

\begin{claim}
Let $F$ be a field and $f, g, h\in F[x]$ are some polynomials.
 Then $(f, g) = (f, g - hf)$.
\end{claim}
\begin{proof}
Indeed, the set of divisors for the pair $\{f, g\}$ is the same as for the pair $\{f, g-hf\}$.
In particular, the maximal elements are also the same, that is the greatest common divisors are the same.
\end{proof}

The latter claim enables us to compute a greatest common divisor in an effective way using Euclidean algorithm.

\paragraph{Input:}

Two polynomials $f, g\in F[x]$.
Here $F$ is a field.

\paragraph{Output:}

A greatest common divisor $d\in F[x]$.

\paragraph{Algorithm}

We use two temporary variables $u, v\in F[x]$.

\begin{enumerate}
\item Initialize $u = f$, $v = g$ in case $\deg f \geqslant \deg g$ and $u = g$, $v = f$ otherwise.

\item While $v \neq 0$ do the following:
\begin{enumerate}
\item Divide $u$ with reminder by $v$ and get $u = q v + r$.

\item Replace $u = v$, $v = r$.
\end{enumerate}

\item When $v = 0$, $u$ becomes a greatest common divisor of the initial $f$ and $g$.
\end{enumerate}


\subsection{Unique Factorization Domain}

\begin{definition}
\begin{itemize}
\item A polynomial $ f\in F[x]\setminus F$ is reducible if there exist $g,h\in F[x]$ such that $f = gh$ and $0<\deg (g) < \deg (f)$ and $0 < \deg(h) < \deg(f)$.

\item A polynomial $ f\in F[x]\setminus F$ is irreducible if for any $g,h\in F[x]$ such that $f = gh$, either $g$ or $h$ is a nonzero constant.
\end{itemize}
\end{definition}

It should be noted that all nonzero polynomials are divided into three classes: 1) invertible polynomials, that is, $F^*$, 2) reducible polynomials, 3) irreducible polynomials.

\begin{claim}
[UFD]
\label{claim::PolyUFD}
Let $F$ be a field.
Then every element $f\in F[x]\setminus F$ is uniquely  presented in the form $f = a p_1^{k_1}\ldots p_n^{k_n}$, where $a\in F$ is a nonzero constant, $k_i$ are positive integers, and $p_i$ are distinct irreducible monic polynomials.
\end{claim}

I do not want to waste our time on proving this result.
An important remark is that the behavior of the polynomials resembles that of integer numbers.
The key argument in the proof of the claim is item~(1) of Claim~\ref{claim::PolyGCD}, that is the fact, that a greatest common divisor of polynomials is a linear combination of the polynomials.
However, it is worth mentioning the following partial case of the claim.

\begin{claim}
Let $F$ be a field, $f, g\in F[x]$ are coprime polynomials, and $h\in F[x]$ is divisible by $f$ and $g$.
Then, $h$ is divisible by $fg$.
\end{claim}
\begin{proof}
Since $f$ and $g$ are coprime, we have $1 = uf + vg$ for some $u,v\in F[x]$ by Claim~\ref{claim::PolyGCD} item~(1).
Multiplying the equality by $h$, we get $h = u hf + v hg$.
Since $g$  divides $h$, $gf$ divides $uhf$.
Since $f$ divides $h$, $fg$ divides $vhg$.
Thus, $fg$ divides $h$.
\end{proof}

\subsection{Ring of remainders}

Now we are ready to meet one of the most important objects in algebra, that is the ring of remainders in case of polynomials.

Let $F$ be a field and $f\in F[x]$ be any polynomial.
I am going to define the ring $F[x]/(f)$.
First, I need to specify a set, then two operations: addition and multiplication, and finally, I should check all the axioms.
If $f = 0$, we define $F[x]/(f)$ to be the polynomial ring itself $F[x]$.
The interesting case is when $f \neq 0$:
\begin{itemize}
\item $F[x]/(f) = \{g \in F[x]\mid \det g < \deg f\}$ the set of reminders with respect to $f$.

\item $+\colon F[x]/(f)\times F[x]/(f) \to F[x]/(f)$ is the usual addition of polynomials.

\item $\cdot \colon F[x]/(f)\times F[x]/(f) \to F[x]/(f)$ is the multiplication modulo $f$, namely: for every $g, h\in F[x]/(f)$, we define $gh \pmod{f}$.
The latter means, we divide $gh$ by $f$ with reminder and get $gh = q f + r$.
Then the product of $g$ and $h$ is $r$.
\end{itemize}

\begin{claim}
If $F$ is a field and $f\in F[x]$ is a polynomial, the set  $F[x]/(f)$ with the given operations is a commutative ring.
\end{claim}
\begin{proof}
If $f = 0$, this is clear because $F[x]/(f) = F[x]$ by definition.
Suppose that $f \neq 0$.
The set $F[x]/(f) = \{g\in F[x]\mid \deg g < \deg f\}$ with addition is an abelian group because its a subgroup in $(F[x], +)$.

Now we need to show: 1) the distributivity law, 2) associativity of multiplication, 3) existence of a neutral element for multiplication, 4) commutativity of multiplication.

1) If $g,h,p\in F[x]/(f)$, we should show that
\[
(g+h)p\!\!\mod{f} = gp\!\!\mod{f} + hp\!\!\mod{f}\quad\text{and}\quad
g(h+p)\!\!\mod{f} = gh\!\!\mod{f} + gp\!\!\mod{f}
\]
We will show the first one.
The second one follows from item~(4).
We divide $gp$ and $hp$ by $f$ with reminder and get
\[
gp = q_1 f + r_1,\;\deg r_1 < \deg f,\;\text{and}\;hp = q_2 f + r_2,\;\deg r_2 < \deg f
\]
How, the right-hand side is $r_1 + r_2$ by definition.
From the other hand the expression
\[
(g+h)p = (q_1 + q_2)f + r_1 + r_2
\]
is a division of $(g+h)p$ by $f$ and the remained here is $r_1 + r_2$.
Hence, the left-hand side is the same.

2) If $g,h,p\in F[x]/(f)$, then 
\[
(g\cdot  (h\cdot  p\!\!\mod{f}))\!\!\mod{f}= (g\cdot  h\cdot  p)\!\!\mod{f} = ((g \cdot h\!\!\mod{f}) \cdot p)\!\!\mod{f}
\]
If am going to leave this as an exercise in abstract nonsense.

3) The polynomial $1$ is the neutral element by definition.

4) The multiplication is commutative by definition.

\end{proof}

\begin{remarks}
\begin{itemize}
\item It should be noted that we may consider $F[x]/(f)$ as a subset (even a subgroup with respect to addition) in $F[x]$.
However, this inclusion does not preserve the multiplication.
The latter means that $F[x]/(f)$ is NOT a subring of $F[x]$.

\item From the other hand, the map $F[x]\to F[x]/(f)$ by the rule $g\mapsto g\!\!\mod{f}$ is a surjective ring homomorphism.
\end{itemize}
\end{remarks}

\begin{claim}
\label{claim::PolyRemIdeals}
Let $F$ be a field, $f\in F[x]$ be a polynomial, and $I\subseteq F[x]/ (f)$ an ideal.
Then there is a polynomial $g\in F[x]$ dividing $f$ such that $I = (g) = \{g h\!\!\mod{f}\mid h\in F[x]\}$.
\end{claim}
\begin{proof}
The case of $f = 0$ is covered in Claim~\ref{claim::PolyIdeals}.
Now we assume that $f\neq 0$.
If $I$ consists of the zero only, then $I = (f)$ and we are done.

Let $h\in I$ be a nonzero polynomial of the least possible degree.
Then for any $g\in I$, we divide $g$ by $h$ with remainders and get $g = qh + r$ and $\deg r < \deg h$.
Also, $r = g - qh\in I$.
Since $h$ was nonzero with least possible degree in $I$, the only option for $r$ is $0$.
Hence, $h$ divides any $g\in I$.
The latter means $I = (h)$.

Now, we need to show that $h$ divides $f$.
Let us divide $f$ by $h$ with remainder, we get $f = qh + r$, where $\deg r < \deg h$.
This means that $r = -qh$ in $F[x]/(f)$.
In particular, $r\in I$ and has degree smaller than $h$.
The only possible option here is $r = 0$ and we are done.
\end{proof}


\begin{claim}
[The Chinese Remainder Theorem]
Let $f, g\in F[x]$ be coprime polynomials, that is $(f, g) = 1$.
Then the map
\[
\Phi\colon F[x]/(fg) \to F[x]/(f)\times F[x]/(g)\quad h \mapsto (h\!\!\mod{f},h\!\!\mod{g})
\]
is an isomorphism of rings.
\end{claim}
\begin{proof}
First, we check that the map is a homomorphism of rings.
We need to show that it preserves the addition, the multiplication and the identity.
This is a straightforward computation and I skip this.

Now, we should show that it is injective.
By Claim~\ref{claim::RingHomProp} it is enough to show that the kernel of $\Phi$ is zero.
Suppose $h\in \ker \Phi$.
This means $h = 0 \pmod f$ and $h = 0 \pmod g$, that is $h$ is divisible by $f$ and $g$.
Since $f$ and $g$ are coprime the latter means that $h$ is divisible by $fg$.
But $\deg h < \deg fg$.
The only possible case is $h = 0$.

Now, we should show the surjectivity.
Since $F$ is a field $F[x]/(fg)$ and $F[x]/(f)\times F[x]/(g)$ are vector spaces over $F$.
Moreover, the map $\Phi$ is a linear map because it preserves the addition and multiplication by any polynomial, hence it preserves the addition and multiplication by any constant.
Since $\Phi$ is injective, it is enough to show that both spaces have the same dimension.
It is clear that $\dim_F F[x]/(fg) = \deg(fg)$.
Also the dimension of the right-hand side is $\deg f + \deg g$.
Now the result follows from Claim~\ref{claim::Degree}.
\end{proof}
